\documentclass{letter}

\begin{document}
Dear Editor, 
We would like to submit our manuscript entitled “Regulation of membrane scission in yeast endocytosis” to be considered for publication as an “Article” in your journal Molecular Biology of the Cell. 
In the process of endocytosis, membrane scission is the final membrane-shaping stage which  results in the formation of a cargo-filled vesicle. In mammalian cells, the force required for scission in clathrin-mediated endocytosis is produced by the GTPase Dynamin in concert with BAR domain proteins. In the only endocytic pathway in the evolutionarily older model system of S.cerevisiae, what causes membrane scission is still unclear. Endocytosis machineries are conserved between yeast and mammalian cells, and endocytic invaginations undergo similar shape transitions from flat membrane to spherical vesicles. Differences in the scission machinery may have come about to accommodate different force requirements required for pulling and severing the membrane in these cell types. 
In this manuscript we have studied the different proteins implicated in yeast scission: yeast N-BAR domain protein complex Rvs167/161 (Rvs), dynamin-like Vps1, and synaptojanin-like proteins Inp51, Inp52, and Inp53. We have used live-cell imaging and genetic manipulation to elucidate the roles of these proteins in membrane scission. We found that the Rvs complex has the largest influence on scission efficiency and the timing of scission. We further studied the recruitment of Rvs and probed the mechanism by which it may regulate scission timing. We show for the first time in vivo that the Rvs BAR domain requires membrane curvature to arrive at endocytic sites. We found that apart from the BAR domain, the SH3 domain of Rvs plays an important role in recruiting Rvs to endocytic sites. The SH3 domain aid the recruitment of BAR domains to membrane curvature, and we suggest that this BAR-membrane interaction stabilizes membrane invagination and delays scission. We finally propose that the force required for scission may come from a different component of the endocytic pathway, the actin network. 

\end{document}