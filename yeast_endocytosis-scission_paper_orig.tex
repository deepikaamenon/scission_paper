Automatically generated by Mendeley Desktop 1.19.4

Any changes to this file will be lost if it is regenerated by Mendeley.



BibTeX export options can be customized via Preferences -> BibTeX in Mendeley Desktop



@article{Smaczynska-deRooij2012,

abstract = {Dynamins are a conserved family of proteins involved in many membrane fusion and fission events. Previously, the dynamin-related protein Vps1 was shown to localize to endocytic sites, and yeast carrying deletions for genes encoding both the BAR domain protein Rvs167 and Vps1 had a more severe endocytic scission defect than either deletion alone. Vps1 and Rvs167 localize to endocytic sites at the onset of invagination and disassemble concomitant with inward vesicle movement. Rvs167-GFP localization is reduced in cells lacking vps1 suggesting that Vps1 influences Rvs167 association with the endocytic complex. Unlike classical dynamins, Vps1 does not have a proline–arginine domain that could interact with SH3 domain-containing proteins. Thus, while Rvs167 has an SH3 domain, it is not clear how an interaction would be mediated. Here, we demonstrate an interaction between Rvs167 SH3 domain and the single type ISH3-binding motif in Vps1. Mutant Vps1 that cannot bind Rvs167 rescues all membrane fusion/fission functions associated with Vps1 except for endocytic function, demonstrating the specificity and mechanistic importance of the interaction. In vitro, an Rvs161/Rvs167 heterodimer can disassemble Vps1 oligomers. Overall, the data support the idea that Vps1 and the amphiphysins function together to mediate scission during endocytosis in yeast.},

author = {{Smaczynska-de Rooij}, Iwona I. and Allwood, Ellen G. and Mishra, Ritu and Booth, Wesley I. and Aghamohammadzadeh, Soheil and Goldberg, Martin W. and Ayscough, Kathryn R.},

doi = {10.1111/j.1600-0854.2011.01311.x},

file = {:Users/deepikaa/Library/Application Support/Mendeley Desktop/Downloaded/Smaczynska-de Rooij et al. - 2012 - Yeast Dynamin Vps1 and Amphiphysin Rvs167 Function Together During Endocytosis.pdf:pdf;:Users/deepikaa/Library/Application Support/Mendeley Desktop/Downloaded/TCKSMRJV/abstract$\backslash$;jsessionid=ECAC7FD488B4FF3704D0E8F5AB6B5503.html:html},

issn = {1600-0854},

journal = {Traffic},

keywords = {3 domain,BAR domain protein,Folder - bar proteins,SH,Saccharomyces cerevisiae,membrane curvature,scission},

language = {en},

mendeley-tags = {3 domain,BAR domain protein,Folder - bar proteins,SH,Saccharomyces cerevisiae,membrane curvature,scission},

pages = {317--328},

title = {{Yeast Dynamin Vps1 and Amphiphysin Rvs167 Function Together During Endocytosis}},

url = {http://onlinelibrary.wiley.com/doi/10.1111/j.1600-0854.2011.01311.x/abstract http://onlinelibrary.wiley.com/doi/10.1111/j.1600-0854.2011.01311.x/abstract;jsessionid=ECAC7FD488B4FF3704D0E8F5AB6B5503.f03t02 http://onlinelibrary.wiley.com/store/10.1111/j.160},

volume = {13},

year = {2012}

}

@article{Ferguson2009,

abstract = {The GTPase dynamin, a key player in endocytic membrane fission, interacts with numerous proteins that regulate actin dynamics and generate/sense membrane curvature. To determine the functional relationship between these proteins and dynamin, we have analyzed endocytic intermediates that accumulate in cells that lack dynamin (derived from dynamin 1 and 2 double conditional knockout mice). In these cells, actin-nucleating proteins, actin, and BAR domain proteins accumulate at the base of arrested endocytic clathrin-coated pits, where they support the growth of dynamic long tubular necks. These results, which we show reflect the sequence of events in wild-type cells, demonstrate a concerted action of these proteins prior to, and independent of, dynamin and emphasize similarities between clathrin-mediated endocytosis in yeast and higher eukaryotes. Our data also demonstrate that the relationship between dynamin and actin is intimately connected to dynamin's endocytic role and that dynamin terminates a powerful actin- and BAR protein-dependent tubulating activity.},

author = {Ferguson, Shawn M and Raimondi, Andrea and Paradise, Summer and Shen, Hongying and Mesaki, Kumi and Ferguson, Agnes and Destaing, Olivier and Ko, Genevieve and Takasaki, Junko and Cremona, Ottavio and {O' Toole}, Eileen and {De Camilli}, Pietro},

doi = {10.1016/j.devcel.2009.11.005},

issn = {1878-1551},

journal = {Developmental cell},

keywords = {Actins,Animals,Cell Membrane,Clathrin,Coated Pits- Cell-Membrane,Dynamin I,Dynamin II,Folder - bar proteins,Gene Knockout Techniques,Mice,Mice- Knockout,cytoskeleton,endocytosis,mammalian endocytosis},

language = {eng},

mendeley-tags = {Actins,Animals,Cell Membrane,Clathrin,Coated Pits- Cell-Membrane,Dynamin I,Dynamin II,Folder - bar proteins,Gene Knockout Techniques,Mice,Mice- Knockout,cytoskeleton,endocytosis,mammalian endocytosis},

month = {dec},

pages = {811--822},

title = {{Coordinated actions of actin and BAR proteins upstream of dynamin at endocytic clathrin-coated pits}},

url = {http://www.ncbi.nlm.nih.gov/pubmed/20059951},

volume = {17},

year = {2009}

}

@article{Kaksonen2005,

abstract = {Endocytosis depends on an extensive network of interacting proteins that execute a series of distinct subprocesses. Previously, we used live-cell imaging of six budding-yeast proteins to define a pathway for association of receptors, adaptors, and actin during endocytic internalization. Here, we analyzed the effects of 61 deletion mutants on the dynamics of this pathway, revealing functions for 15 proteins, and we analyzed the dynamics of 8 of these proteins. Our studies provide evidence for four protein modules that cooperate to drive coat formation, membrane invagination, actin-meshwork assembly, and vesicle scission during clathrin/actin-mediated endocytosis. We found that clathrin facilitates the initiation of endocytic-site assembly but is not needed for membrane invagination or vesicle formation. Finally, we present evidence that the actin-meshwork assembly that drives membrane invagination is nucleated proximally to the plasma membrane, opposite to the orientation observed for previously studied actin-assembly-driven motility processes. Copyright {\textcopyright}2005 by Elsevier Inc.},

author = {Kaksonen, Marko and Toret, Christopher P. and Drubin, David G.},

doi = {10.1016/j.cell.2005.09.024},

file = {:Users/deepikaa/Library/Application Support/Mendeley Desktop/Downloaded/K9EI5APA/reload=0$\backslash$;jsessionid=eduACLQWvI23hGgNXfxi.html:html},

issn = {00928674},

journal = {Cell},

keywords = {Folder - review},

mendeley-tags = {Folder - review},

month = {oct},

pages = {305--320},

pmid = {16239147},

title = {{A modular design for the clathrin- and actin-mediated endocytosis machinery}},

url = {http://europepmc.org/abstract/MED/16239147/reload=0;jsessionid=eduACLQWvI23hGgNXfxi.8},

volume = {123},

year = {2005}

}

@article{Madania1999,

abstract = {Yeast Las17 protein is homologous to the Wiskott–Aldrich Syndrome protein, which is implicated in severe immunodeficiency. Las17p/Bee1p has been shown to be important for actin patch assembly and actin polymerization. Here we show that Las17p interacts with the Arp2/3 complex. LAS17 is an allele-specific multicopy suppressor of ARP2 and ARP3 mutations; overexpression restores both actin patch organization and endocytosis defects in ARP2 temperature-sensitive (ts) cells. Six of seven ARP2 ts mutants and at least one ARP3 ts mutant are synthetically lethal with las17$\Delta$ ts confirming functional interaction with the Arp2/3 complex. Further characterization of las17$\Delta$ cells showed that receptor-mediated internalization of $\alpha$ factor by the Ste2 receptor is severely defective. The polarity of normal bipolar bud site selection is lost. Las17-gfp remains localized in cortical patches in vivo independently of polymerized actin and is required for the polarized localization of Arp2/3 as well as actin. Coimmunoprecipitation of Arp2p with Las17p indicates that Las17p interacts directly with the complex. Two hybrid results also suggest that Las17p interacts with actin, verprolin, Rvs167p and several other proteins including Src homology 3 (SH3) domain proteins, suggesting that Las17p may integrate signals from different regulatory cascades destined for the Arp2/3p complex and the actin cytoskeleton.},

author = {Madania, Ammar and Dumoulin, Pascal and Grava, Sandrine and Kitamoto, Hiroko and Scharer-Brodbeck, Claudia and Soulard, Alexandre and Moreau, Violaine and Winsor, Barbara},

file = {:Users/deepikaa/Library/Application Support/Mendeley Desktop/Downloaded/Madania et al. - 1999 - The Saccharomyces cerevisiae homologue of human Wiskott-Aldrich syndrome protein Las17p interacts with the Arp23.pdf:pdf},

issn = {1059-1524},

journal = {Molecular Biology of the Cell},

keywords = {Folder - other proteins},

mendeley-tags = {Folder - other proteins},

month = {oct},

pages = {3521--3538},

title = {{The Saccharomyces cerevisiae Homologue of Human Wiskott-Aldrich Syndrome Protein Las17p Interacts with the Arp2/3 Complex}},

url = {http://www.ncbi.nlm.nih.gov/pmc/articles/PMC25621/ http://www.ncbi.nlm.nih.gov/pmc/articles/PMC25621/pdf/mk003521.pdf},

volume = {10},

year = {1999}

}

@article{Heuser1973,

abstract = {When the nerves of isolated frog sartorius muscles were stimulated at 10 Hz, synaptic vesicles in the motor nerve terminals became transiently depleted. This depletion apparently resulted from a redistribution rather than disappearance of synaptic vesicle membrane, since the total amount of membrane comprising these nerve terminals remained constant during stimulation. At 1 min of stimulation, the 30{\%} depletion in synaptic vesicle membrane was nearly balanced by an increase in plasma membrane, suggesting that vesicle membrane rapidly moved to the surface as it might if vesicles released their content of transmitter by exocytosis. After 15 min of stimulation, the 60{\%} depletion of synaptic vesicle membrane was largely balanced by the appearance of numerous irregular membrane-walled cisternae inside the terminals, suggesting that vesicle membrane was retrieved from the surface as cisternae. When muscles were rested after 15 min of stimulation, cisternae disappeared and synaptic vesicles reappeared, suggesting that cisternae divided to form new synaptic vesicles so that the original vesicle membrane was now recycled into new synaptic vesicles. When muscles were soaked in horseradish peroxidase (HRP), this tracerfirst entered the cisternae which formed during stimulation and then entered a large proportion of the synaptic vesicles which reappeared during rest, strengthening the idea that synaptic vesicle membrane added to the surface was retrieved as cisternae which subsequently divided to form new vesicles. When muscles containing HRP in synaptic vesicles were washed to remove extracellular HRP and restimulated, HRP disappeared from vesicles without appearing in the new cisternae formed during the second stimulation, confirming that a one-way recycling of synaptic membrane, from the surface through cisternae to new vesicles, was occurring. Coated vesicles apparently represented the actual mechanism for retrieval of synaptic vesicle membrane from the plasma membrane, because during nerve stimulation they proliferated at regions of the nerve terminals covered by Schwann processes, took up peroxidase, and appeared in various stages of coalescence with cisternae. In contrast, synaptic vesicles did not appear to return directly from the surface to form cisternae, and cisternae themselves never appeared directly connected to the surface. Thus, during stimulation the intracellular compartments of this synapse change shape and take up extracellular protein in a manner which indicates that synaptic vesicle membrane added to the surface during exocytosis is retrieved by coated vesicles and recycled into new synaptic vesicles by way of intermediate cisternae.},

author = {Heuser, J E and Reese, T S},

issn = {0021-9525},

journal = {The Journal of cell biology},

month = {may},

pages = {315--44},

pmid = {4348786},

title = {{Evidence for recycling of synaptic vesicle membrane during transmitter release at the frog neuromuscular junction.}},

url = {http://www.ncbi.nlm.nih.gov/pubmed/4348786 http://www.pubmedcentral.nih.gov/articlerender.fcgi?artid=PMC2108984},

volume = {57},

year = {1973}

}

@article{Skruzny2012,

abstract = {Dynamic actin filaments are a crucial component of clathrin-mediated endocytosis when endocytic proteins cannot supply enough energy for vesicle budding. Actin cytoskeleton is thought to provide force for membrane invagination or vesicle scission, but how this force is transmitted to the plasma membrane is not understood. Here we describe the molecular mechanism of plasma membrane-actin cytoskeleton coupling mediated by cooperative action of epsin Ent1 and the HIP1R homolog Sla2 in yeast Saccharomyces cerevisiae. Sla2 anchors Ent1 to a stable endocytic coat by an unforeseen interaction between Sla2's ANTH and Ent1's ENTH lipid-binding domains. The ANTH and ENTH domains bind each other in a ligand-dependent manner to provide critical anchoring of both proteins to the membrane. The C-terminal parts of Ent1 and Sla2 bind redundantly to actin filaments via a previously unknown phospho-regulated actin-binding domain in Ent1 and the THATCH domain in Sla2. By the synergistic binding to the membrane and redundant interaction with actin, Ent1 and Sla2 form an essential molecular linker that transmits the force generated by the actin cytoskeleton to the plasma membrane, leading to membrane invagination and vesicle budding.},

author = {Skruzny, Michal and Brach, Thorsten and Ciuffa, Rodolfo and Rybina, Sofia and Wachsmuth, Malte and Kaksonen, Marko},

doi = {10.1073/pnas.1207011109},

file = {:Users/deepikaa/Library/Application Support/Mendeley Desktop/Downloaded/Skruzny et al. - 2012 - Molecular basis for coupling the plasma membrane to the actin cytoskeleton during clathrin-mediated endocytosis.pdf:pdf},

issn = {1091-6490},

journal = {Proceedings of the National Academy of Sciences of the United States of America},

month = {sep},

pages = {15092-15093},

pmid = {22927393},

publisher = {National Academy of Sciences},

title = {{Molecular basis for coupling the plasma membrane to the actin cytoskeleton during clathrin-mediated endocytosis.}},

url = {http://www.ncbi.nlm.nih.gov/pubmed/22927393 http://www.pubmedcentral.nih.gov/articlerender.fcgi?artid=PMC3458359},

volume = {109},

year = {2012}

}

@article{Bui2012,

abstract = {To initiate mitochondrial fission, dynamin-related proteins (DRPs) must bind specific adaptors on the outer mitochondrial membrane. The structural features underlying this interaction are poorly understood. Using yeast as a model, we show that the Insert B domain of the Dnm1 guanosine triphosphatase (a DRP) contains a novel motif required for association with the mitochondrial adaptor Mdv1. Mutation of this conserved motif specifically disrupted Dnm1-Mdv1 interactions, blocking Dnm1 recruitment and mitochondrial fission. Suppressor mutations in Mdv1 that restored Dnm1-Mdv1 interactions and fission identified potential protein-binding interfaces on the Mdv1 $\beta$-propeller domain. These results define the first known function for Insert B in DRP-adaptor interactions. Based on the variability of Insert B sequences and adaptor proteins, we propose that Insert B domains and mitochondrial adaptors have coevolved to meet the unique requirements for mitochondrial fission of different organisms.},

author = {Bui, Huyen T and Karren, Mary A and Bhar, Debjani and Shaw, Janet M},

doi = {10.1083/jcb.201207079},

file = {:Users/deepikaa/Library/Application Support/Mendeley Desktop/Downloaded/Bui et al. - 2012 - A novel motif in the yeast mitochondrial dynamin Dnm1 is essential for adaptor binding and membrane recruitment.pdf:pdf},

issn = {1540-8140},

journal = {The Journal of cell biology},

month = {nov},

pages = {613--22},

pmid = {23148233},

publisher = {Rockefeller University Press},

title = {{A novel motif in the yeast mitochondrial dynamin Dnm1 is essential for adaptor binding and membrane recruitment.}},

url = {http://www.ncbi.nlm.nih.gov/pubmed/23148233 http://www.pubmedcentral.nih.gov/articlerender.fcgi?artid=PMC3494853},

volume = {199},

year = {2012}

}

@article{DHondt2000,

abstract = {▪ Abstract Genetic and biochemical studies in yeast and animal cells have led to the identification of many components required for endocytosis. In this review, we summarize our understanding of the endocytic machinery with an emphasis on the proteins regulating the internalization step of endocytosis and endosome fusion. Even though the overall endocytic machinery appears to be conserved between yeast and animals, clear differences exist. We also discuss the roles of phosphoinositides, sterols, and sphingolipid precursors in endocytosis, because in addition to proteins, these lipids have emerged as important determinants in the spatial and most likely temporal specificity of endocytic membrane trafficking events.},

author = {D'Hondt, Kathleen and Heese-Peck, Antje and Riezman, Howard},

doi = {10.1146/annurev.genet.34.1.255},

issn = {0066-4197},

journal = {Annual Review of Genetics},

keywords = {EH-domain,actin,clathrin,dynamin,endosome,phosphoinositides,sterol},

month = {dec},

pages = {255--295},

publisher = { Annual Reviews  4139 El Camino Way, P.O. Box 10139, Palo Alto, CA 94303-0139, USA  },

title = {{Protein and Lipid Requirements for Endocytosis}},

url = {http://www.annualreviews.org/doi/10.1146/annurev.genet.34.1.255},

volume = {34},

year = {2000}

}

@article{Boeke2014,

abstract = {Clathrin-mediated endocytosis is a highly conserved intracellular trafficking pathway that depends on dynamic protein-protein interactions between up to 60 different proteins. However, little is known about the spatio-temporal regulation of these interactions. Using fluorescence (cross)-correlation spectroscopy in yeast, we tested 41 previously reported interactions in vivo and found 16 to exist in the cytoplasm. These detected cytoplasmic interactions included the self-interaction of Ede1, homolog of mammalian Eps15. Ede1 is the crucial scaffold for the organization of the early stages of endocytosis. We show that oligomerization of Ede1 through its central coiled coil domain is necessary for its localization to the endocytic site and we link the oligomerization of Ede1 to its function in locally concentrating endocytic adaptors and organizing the endocytic machinery. Our study sheds light on the importance of the regulation of protein-protein interactions in the cytoplasm for the assembly of the endocytic machinery in vivo.},

author = {Boeke, Dominik and Trautmann, Susanne and Meurer, Matthias and Wachsmuth, Malte and Godlee, Camilla and Knop, Michael and Kaksonen, Marko},

doi = {10.15252/msb.20145422},

file = {:Users/deepikaa/Library/Application Support/Mendeley Desktop/Downloaded/Boeke et al. - 2014 - Quantification of cytosolic interactions identifies Ede1 oligomers as key organizers of endocytosis.pdf:pdf},

issn = {1744-4292},

journal = {Molecular Systems Biology},

keywords = {Ede1,endocytosis,fluorescence (cross‐)correlation spectroscopy},

month = {nov},

pages = {756},

publisher = {EMBO},

title = {{Quantification of cytosolic interactions identifies Ede1 oligomers as key organizers of endocytosis}},

url = {https://onlinelibrary.wiley.com/doi/abs/10.15252/msb.20145422},

volume = {10},

year = {2014}

}

@article{Liu2006,

abstract = {Endocytosis in budding yeast is thought to occur in several phases. First, the membrane invaginates and then elongates into a tube. A vesicle forms at the end of the tube, eventually pinching off to form a "free" vesicle. Experiments show that actin polymerization is an active participant in the endocytic process, along with a number of membrane-associated proteins. Here we investigate the possible roles of these components in driving vesiculation by constructing a quantitative model of the process beginning at the stage where the membrane invagination has elongated into a tube encased in a sheath of membrane-associated protein. This protein sheath brings about the scission step where the vesicle separates from the tube. When the protein sheath is dynamin, it is commonly assumed that scission is brought about by the constriction of the sheath. Here, we show that an alternative scenario can work as well: The protein sheath acts as a "filter" to effect a phase separation of lipid species. The resulting line tension tends to minimize the interface between the tube region and the vesicle region. Interestingly, large vesicle size can further facilitate the reduction of the interfacial diameter down to a few nanometers, small enough so that thermal fluctuations can fuse the membrane and pinch off the vesicle. To deform the membrane into the tubular vesicle shape, the membrane elastic resistance forces must be balanced by some additional forces that we show can be generated by actin polymerization and/or myosin I. These active forces are shown to be important in successful scission processes as well.},

author = {Liu, Jian and Kaksonen, Marko and Drubin, David G and Oster, George},

doi = {10.1073/pnas.0601045103},

file = {:Users/deepikaa/Library/Application Support/Mendeley Desktop/Downloaded/Liu et al. - 2006 - Endocytic vesicle scission by lipid phase boundary forces(2).pdf:pdf},

issn = {0027-8424},

journal = {Proceedings of the National Academy of Sciences of the United States of America},

month = {jul},

pages = {10277--82},

pmid = {16801551},

publisher = {National Academy of Sciences},

title = {{Endocytic vesicle scission by lipid phase boundary forces.}},

url = {http://www.ncbi.nlm.nih.gov/pubmed/16801551 http://www.pubmedcentral.nih.gov/articlerender.fcgi?artid=PMC1502448},

volume = {103},

year = {2006}

}

@article{Kaksonen2003,

abstract = {In budding yeast, many proteins involved in endocytic internalization, including adaptors and actin cytoskeletal proteins, are localized to cortical patches of differing protein composition. Using multicolor real-time fluorescence microscopy and particle tracking algorithms, we define an early endocytic pathway wherein an invariant sequence of changes in cortical patch protein composition correlates with changes in patch motility. Three Arp2/3 activators each showed a distinct behavior, suggesting distinct patch-related endocytic functions. Actin polymerization occurs late in the endocytic pathway and is required both for endocytic internalization and for patch disassembly. In cells lacking the highly conserved endocytic protein Sla2p, patch motility was arrested and actin comet tails associated with endocytic patch complexes. Fluorescence recovery after photobleaching of the actin comet tails revealed that endocytic complexes are nucleation sites for rapid actin polymerization. Attention is now focused on the mechanisms by which the order and timing of events in this endocytic pathway are achieved.},

author = {Kaksonen, Marko and Sun, Yidi and Drubin, David G.},

issn = {0092-8674},

journal = {Cell},

keywords = {Actins,Algorithms,Carrier Proteins,Cytoskeletal Proteins,Folder - now,Microscopy- Fluorescence,Receptors- Cell Surface,Receptors- Mating Factor,Receptors- Peptide,Saccharomyces cerevisiae,Saccharomyces cerevisiae Proteins,Transcription Factors,Transport Vesicles,Two-Hybrid System Techniques,endocytosis},

language = {ENG},

mendeley-tags = {Actins,Algorithms,Carrier Proteins,Cytoskeletal Proteins,Folder - now,Microscopy- Fluorescence,Receptors- Cell Surface,Receptors- Mating Factor,Receptors- Peptide,Saccharomyces cerevisiae,Saccharomyces cerevisiae Proteins,Transcription Factors,Transport Vesicles,Two-Hybrid System Techniques,endocytosis},

month = {nov},

pages = {475--487},

title = {{A pathway for association of receptors, adaptors, and actin during endocytic internalization}},

url = {http://www.ncbi.nlm.nih.gov/pubmed/14622601},

volume = {115},

year = {2003}

}

@article{Moustaq2016,

abstract = {The dynamins represent a superfamily of proteins that have been shown to function in a wide range of membrane fusion and fission events. An increasing number of mutations in the human classical dynamins, Dyn-1 and Dyn-2 has been reported, with diseases caused by these changes ranging from Charcot-Marie-Tooth disorder to epileptic encephalopathies. The budding yeast, Saccharomyces cerevisiae expresses a single dynamin-related protein that functions in membrane trafficking, and is considered to play a similar role to Dyn-1 and Dyn-2 during scission of endocytic vesicles at the plasma membrane. Large parts of the dynamin protein are highly conserved across species and this has enabled us in this study to select a number of disease causing mutations and to generate equivalent mutations in Vps1. We have then studied these mutants using both cellular and biochemical assays to ascertain functions of the protein that have been affected by the changes. Specifically, we demonstrate that the Vps1-G397R mutation (Dyn-2 G358R) disrupts protein oligomerization, Vps1-A447T (Dyn-1 A408T) affects the scission stage of endocytosis, while Vps1-R298L (Dyn-1 R256L) affects lipid binding specificity and possibly an early stage in endocytosis. Overall, we consider that the yeast model will potentially provide an avenue for rapid analysis of new dynamin mutations in order to understand the underlying mechanisms that they disrupt.},

author = {Moustaq, Laila and {Smaczynska-de Rooij}, Iwona I and Palmer, Sarah E and Marklew, Christopher J and Ayscough, Kathryn R},

doi = {10.15698/mic2016.04.490},

file = {:Users/deepikaa/Library/Application Support/Mendeley Desktop/Downloaded/Moustaq et al. - 2016 - Insights into dynamin-associated disorders through analysis of equivalent mutations in the yeast dynamin Vps1.pdf:pdf},

issn = {2311-2638},

journal = {Microbial cell},

keywords = {Charcot-Marie-Tooth,Disease mutation,Dynamin,Epilepsy,Saccharomyces cerevisiae},

month = {mar},

pages = {147--158},

pmid = {28357347},

publisher = {Shared Science Publishers},

title = {{Insights into dynamin-associated disorders through analysis of equivalent mutations in the yeast dynamin Vps1.}},

url = {http://www.ncbi.nlm.nih.gov/pubmed/28357347 http://www.pubmedcentral.nih.gov/articlerender.fcgi?artid=PMC5349089},

volume = {3},

year = {2016}

}

@article{Nicot2007,

abstract = {Centronuclear myopathies are characterized by muscle weakness and abnormal centralization of nuclei in muscle fibers not secondary to regeneration. The severe neonatal X-linked form (myotubular myopathy) is due to mutations in the phosphoinositide phosphatase myotubularin (MTM1), whereas mutations in dynamin 2 (DNM2) have been found in some autosomal dominant cases. By direct sequencing of functional candidate genes, we identified homozygous mutations in amphiphysin 2 (BIN1) in three families with autosomal recessive inheritance. Two missense mutations affecting the BAR (Bin1/amphiphysin/RVS167) domain disrupt its membrane tubulation properties in transfected cells, and a partial truncation of the C-terminal SH3 domain abrogates the interaction with DNM2 and its recruitment to the membrane tubules. Our results suggest that mutations in BIN1 cause centronuclear myopathy by interfering with remodeling of T tubules and/or endocytic membranes, and that the functional interaction between BIN1 and DNM2 is necessary for normal muscle function and positioning of nuclei.},

author = {Nicot, Anne-Sophie and Toussaint, Anne and Tosch, Val{\'{e}}rie and Kretz, Christine and Wallgren-Pettersson, Carina and Iwarsson, Erik and Kingston, Helen and Garnier, Jean-Marie and Biancalana, Val{\'{e}}rie and Oldfors, Anders and Mandel, Jean-Louis and Laporte, Jocelyn},

doi = {10.1038/ng2086},

file = {:Users/deepikaa/Library/Application Support/Mendeley Desktop/Downloaded/HI76ZPMH/Nicot et al. - 2007 - Mutations in amphiphysin 2 (BIN1) disrupt interact.pdf:pdf;:Users/deepikaa/Library/Application Support/Mendeley Desktop/Downloaded/M5EUDXS5/ng2086.html:html},

issn = {1061-4036},

journal = {Nature Genetics},

keywords = {Folder - bar proteins},

language = {en},

mendeley-tags = {Folder - bar proteins},

month = {sep},

pages = {1134--1139},

title = {{Mutations in amphiphysin 2 (BIN1) disrupt interaction with dynamin 2 and cause autosomal recessive centronuclear myopathy}},

url = {http://www.nature.com/ng/journal/v39/n9/full/ng2086.html http://www.nature.com/ng/journal/v39/n9/pdf/ng2086.pdf},

volume = {39},

year = {2007}

}

@article{Dmitrieff2015,

abstract = {Author Summary Cells use endocytosis to intake molecules and to recycle components of their membrane. Even in its simplest form, endocytosis involves a large number of proteins with often redundant functions that are organized into a microscopic force-producing “machine”. Knowing how much force is needed to induce a membrane invagination is essential to understand how this endocytic machine may operate. We show that experimental membrane shapes are well described theoretically by a thin sheet elastic model including a difference of pressure across the membrane due to turgor. This allows us to integrate the different contributions that shape the membrane, and to compute the forces opposing membrane deformation. This calculation provides an estimate of the pulling force that must be generated by the actin machinery in yeast. We also identify a membrane instability that could lead to vesicle budding.},

author = {Dmitrieff, Serge and N{\'{e}}d{\'{e}}lec, Fran{\c{c}}ois},

doi = {10.1371/journal.pcbi.1004538},

journal = {PLoS Comput Biol},

keywords = {Folder - theory},

mendeley-tags = {Folder - theory},

month = {oct},

pages = {e1004538},

title = {{Membrane Mechanics of Endocytosis in Cells with Turgor}},

url = {http://dx.doi.org/10.1371/journal.pcbi.1004538 http://www.ploscompbiol.org/article/fetchObject.action?uri=info{\%}3Adoi{\%}2F10.1371{\%}2Fjournal.pcbi.1004538{\&}representation=PDF},

volume = {11},

year = {2015}

}

@article{Hoepfner2001,

author = {Hoepfner, Dominic and van den Berg, Marlene and Philippsen, Peter and Tabak, Henk F. and Hettema, Ewald H.},

doi = {10.1083/jcb.200107028},

issn = {0021-9525},

journal = {The Journal of Cell Biology},

month = {dec},

pages = {979--990},

title = {{A role for Vps1p, actin, and the Myo2p motor in peroxisome abundance and inheritance in \textit{Saccharomyces cerevisiae}}},

url = {http://www.jcb.org/lookup/doi/10.1083/jcb.200107028},

volume = {155},

year = {2001}

}

@article{Zhang2001,

abstract = {Three-dimensional reconstruction of dynamin in the constricted state},

author = {Zhang, Peijun and Hinshaw, Jenny E.},

doi = {10.1038/ncb1001-922},

file = {:Users/deepikaa/Library/Application Support/Mendeley Desktop/Downloaded/Zhang, Hinshaw - 2001 - Three-dimensional reconstruction of dynamin in the constricted state.pdf:pdf},

issn = {1465-7392},

journal = {Nature Cell Biology},

month = {oct},

pages = {922--926},

publisher = {Nature Publishing Group},

title = {{Three-dimensional reconstruction of dynamin in the constricted state}},

url = {http://www.nature.com/articles/ncb1001-922},

volume = {3},

year = {2001}

}

@article{Munn1995,

author = {Munn, A. L. and Stevenson, B. J. and Geli, M. I. and Riezman, H.},

doi = {10.1091/mbc.6.12.1721},

issn = {1059-1524},

journal = {Molecular Biology of the Cell},

month = {dec},

pages = {1721--1742},

title = {{end5, end6, and end7: Mutations that cause actin delocalization and block the internalization step of endocytosis in Saccharomyces cerevisiae.}},

url = {http://www.molbiolcell.org/cgi/doi/10.1091/mbc.6.12.1721},

volume = {6},

year = {1995}

}

@article{Nannapaneni2010b,

author = {Nannapaneni, Srikant and Wang, Daobing and Jain, Sandhya and Schroeder, Blake and Highfill, Chad and Reustle, Lindsay and Pittsley, Delilah and Maysent, Adam and Moulder, Shawn and McDowell, Ryan and Kim, Kyoungtae},

doi = {10.1016/j.ejcb.2010.02.002},

issn = {01719335},

journal = {European Journal of Cell Biology},

month = {jul},

pages = {499--508},

title = {{The yeast dynamin-like protein Vps1:vps1 mutations perturb the internalization and the motility of endocytic vesicles and endosomes via disorganization of the actin cytoskeleton}},

url = {http://linkinghub.elsevier.com/retrieve/pii/S0171933510000440},

volume = {89},

year = {2010}

}

@article{Zhao2016,

abstract = {Membrane fusion and fission are vital for eukaryotic life. For three decades, it has been proposed that fusion is mediated by fusion between the proximal leaflets of two bilayers (hemi-fusion) to produce a hemi-fused structure, followed by fusion between the distal leaflets, whereas fission is via hemi-fission, which also produces a hemi-fused structure, followed by full fission. This hypothesis remained unsupported owing to the lack of observation of hemi-fusion or hemi-fission in live cells. A competing fusion hypothesis involving protein-lined pore formation has also been proposed. Here we report the observation of a hemi-fused $\Omega$-shaped structure in live neuroendocrine chromaffin cells and pancreatic $\beta$-cells, visualized using confocal and super-resolution stimulated emission depletion microscopy. This structure is generated from fusion pore opening or closure (fission) at the plasma membrane. Unexpectedly, the transition to full fusion or fission is determined by competition between fusion and calcium/dynamin-dependent fission mechanisms, and is notably slow (seconds to tens of seconds) in a substantial fraction of the events. These results provide key missing evidence in support of the hemi-fusion and hemi-fission hypothesis in live cells, and reveal the hemi-fused intermediate as a key structure controlling fusion and fission, as fusion and fission mechanisms compete to determine the transition to fusion or fission.},

author = {Zhao, Wei-Dong and Hamid, Edaeni and Shin, Wonchul and Wen, Peter J and Krystofiak, Evan S and Villarreal, Seth A and Chiang, Hsueh-Cheng and Kachar, Bechara and Wu, Ling-Gang},

doi = {10.1038/nature18598},

issn = {1476-4687},

journal = {Nature},

pages = {548--52},

pmid = {27309816},

publisher = {NIH Public Access},

title = {{Hemi-fused structure mediates and controls fusion and fission in live cells.}},

url = {http://www.ncbi.nlm.nih.gov/pubmed/27309816 http://www.pubmedcentral.nih.gov/articlerender.fcgi?artid=PMC4930626},

volume = {534},

year = {2016}

}

@article{Farsad2001,

abstract = {Endophilin 1 is a presynaptically enriched protein which binds the GTPase dynamin and the polyphosphoinositide phosphatase synptojanin. Perturbation of endophilin function in cell-free systems and in a living synapse has implicated endophilin in endocytic vesicle budding (Ringstad, N., H. Gad, P. Low, G. Di Paolo, L. Brodin, O. Shupliakov, and P. De Camilli. 1999. Neuron. 24:143-154; Schmidt, A., M. Wolde, C. Thiele, W. Fest, H. Kratzin, A.V. Podtelejnikov, W. Witke, W.B. Huttner, and H.D. Soling. 1999. Nature. 401:133-141; Gad, H., N. Ringstad, P. Low, O. Kjaerulff, J. Gustafsson, M. Wenk, G. Di Paolo, Y. Nemoto, J. Crun, M.H. Ellisman, et al. 2000. Neuron. 27:301-312). Here, we show that purified endophilin can directly bind and evaginate lipid bilayers into narrow tubules similar in diameter to the neck of a clathrin-coated bud, providing new insight into the mechanisms through which endophilin may participate in membrane deformation and vesicle budding. This property of endophilin is independent of its putative lysophosphatydic acid acyl transferase activity, is mediated by its NH2-terminal region, and requires an amino acid stretch homologous to a corresponding region in amphiphysin, a protein previously shown to have similar effects on lipid bilayers (Takei, K., V.I. Slepnev, V. Haucke, and P. De Camilli. 1999. Nat. Cell Biol. 1:33-39). Endophilin cooligomerizes with dynamin rings on lipid tubules and inhibits dynamin's GTP-dependent vesiculating activity. Endophilin B, a protein with homology to endophilin 1, partially localizes to the Golgi complex and also deforms lipid bilayers into tubules, underscoring a potential role of endophilin family members in diverse tubulovesicular membrane-trafficking events in the cell.},

author = {Farsad, K. and Ringstad, N. and Takei, K. and Floyd, S. R. and Rose, K. and {De Camilli}, P.},

doi = {10.1083/jcb.200107075},

issn = {0021-9525},

journal = {The Journal of Cell Biology},

keywords = {Acyltransferases,Adaptor Proteins- Signal Transducing,Amino Acid Sequence,Animals,Biological Transport,Carrier Proteins,Cell Membrane,Cell Size,Dynamins,Folder - bar proteins,GTP Phosphohydrolases,Golgi Apparatus,Humans,Lipid Bilayers,Macromolecular Substances,Molecular Sequence Data,Nerve Tissue Proteins,Phylogeny,Protein Structure- Tertiary,Rats,Sequence Homology- Amino Acid,Synaptic Vesicles},

language = {eng},

mendeley-tags = {Acyltransferases,Adaptor Proteins- Signal Transducing,Amino Acid Sequence,Animals,Biological Transport,Carrier Proteins,Cell Membrane,Cell Size,Dynamins,Folder - bar proteins,GTP Phosphohydrolases,Golgi Apparatus,Humans,Lipid Bilayers,Macromolecular Substances,Molecular Sequence Data,Nerve Tissue Proteins,Phylogeny,Protein Structure- Tertiary,Rats,Sequence Homology- Amino Acid,Synaptic Vesicles},

month = {oct},

pages = {193--200},

title = {{Generation of high curvature membranes mediated by direct endophilin bilayer interactions}},

url = {http://www.ncbi.nlm.nih.gov/pubmed/11604418},

volume = {155},

year = {2001}

}

@article{Hohendahl2017,

abstract = {{\textless}p{\textgreater}Dynamin, which mediates membrane fission during endocytosis, binds endophilin and other members of the Bin-Amphiphysin-Rvs (BAR) protein family. How endophilin influences endocytic membrane fission is still unclear. Here we show that dynamin-mediated membrane fission is potently inhibited in vitro when an excess of endophilin co-assembles with dynamin around membrane tubules. We further show by electron microscopy that endophilin intercalates between turns of the dynamin helix and impairs fission by preventing trans interactions between dynamin rungs that are thought to play critical roles in membrane constriction. In living cells, overexpression of endophilin delayed both fission and transferrin uptake. Together, our observations suggest that while endophilin helps shape endocytic tubules and recruit dynamin to endocytic sites, it can also block membrane fission when present in excess by inhibiting inter-dynamin interactions. The sequence of recruitment and the relative stoichiometry of the two proteins may be critical to regulated endocytic fission.{\textless}/p{\textgreater}},

author = {Hohendahl, Annika and Talledge, Nathaniel and Galli, Valentina and Shen, Peter S and Humbert, Fr{\'{e}}d{\'{e}}ric and {De Camilli}, Pietro and Frost, Adam and Roux, Aur{\'{e}}lien},

doi = {10.7554/eLife.26856},

issn = {2050-084X},

journal = {eLife},

month = {sep},

title = {{Structural inhibition of dynamin-mediated membrane fission by endophilin}},

url = {https://elifesciences.org/articles/26856},

volume = {6},

year = {2017}

}

@article{ROSEN1978,

author = {ROSEN, ROBERT},

doi = {10.1080/03081077808960692},

issn = {0308-1079},

journal = {International Journal of General Systems},

month = {jan},

pages = {266--269},

publisher = { Taylor {\&} Francis Group },

title = {{Review of: “SELF-ORGANIZATION IN NONEQUILIBRIUM SYSTEMS”, by G. Nicolis and I. Prigogine, John Wiley, New York, 1977, 491 pp.}},

url = {http://www.tandfonline.com/doi/abs/10.1080/03081077808960692},

volume = {4},

year = {1978}

}

@article{Kubler1993a,

abstract = {In Saccharomyces cerevisiae, alpha-factor is internalized by receptor-mediated endocytosis and transported via vesicular intermediates to the vacuole where the pheromone is degraded. Using beta-tubulin and actin mutant strains, we showed that actin plays a direct role in receptor-mediated internalization of alpha-factor, but is not necessary for transport from the endocytic intermediates to the vacuole. beta-tubulin mutant strains showed no defect in these processes. In addition, cells lacking the actin-binding protein, Sac6p, which is the yeast fimbrin homologue, are defective for internalization of alpha-factor suggesting that actin filament bundling might be required for this step. The actin dependence of endocytosis shows some interesting similarities to endocytosis from the apical membrane in polarized mammalian cells.},

author = {K{\"{u}}bler, E and Riezman, H and Riezman, H. and Riezman, Howard},

doi = {10.1093/emboj/17.3.635},

file = {:Users/deepikaa/Library/Application Support/Mendeley Desktop/Downloaded/K{\"{u}}bler et al. - 1993 - Actin and fimbrin are required for the internalization step of endocytosis in yeast.pdf:pdf},

issn = {0261-4189},

journal = {The EMBO journal},

month = {jul},

pages = {2855--62},

pmid = {8335001},

publisher = {EMBO Press},

title = {{Actin and fimbrin are required for the internalization step of endocytosis in yeast.}},

url = {http://www.ncbi.nlm.nih.gov/pubmed/8335001 http://www.pubmedcentral.nih.gov/articlerender.fcgi?artid=PMC413538},

volume = {12},

year = {1993}

}

@article{Takei1995,

abstract = {Tubular membrane invaginations coated by dynamin rings are induced by GTP-$\gamma$S in nerve terminals},

author = {Takei, Kohji and McPherson, Peter S. and Schmid, Sandra L. and Camilli, Pietro De},

doi = {10.1038/374186a0},

file = {:Users/deepikaa/Library/Application Support/Mendeley Desktop/Downloaded/Takei et al. - 1995 - Tubular membrane invaginations coated by dynamin rings are induced by GTP-$\gamma$S in nerve terminals(3).pdf:pdf},

issn = {0028-0836},

journal = {Nature},

month = {mar},

pages = {186--190},

publisher = {Nature Publishing Group},

title = {{Tubular membrane invaginations coated by dynamin rings are induced by GTP-$\gamma$S in nerve terminals}},

url = {http://www.nature.com/doifinder/10.1038/374186a0},

volume = {374},

year = {1995}

}

@article{Nkosi2013,

abstract = {The Hof1 protein (Homologue of Fifteen) regulates formation of the primary septum during cytokinesis in the budding yeast Saccharomyces cerevisiae, whereas the orthologous Cdc15 protein in fission yeast regulates the actomyosin ring by using its F-BAR domain to recruit actin nucleators to the cleavage site. Here we show that budding yeast Hof1 also contributes to actin ring assembly in parallel with the Rvs167 protein. Simultaneous deletion of the HOF1 and RVS167 genes is lethal, and cells fail to assemble the actomyosin ring as they progress through mitosis. Although Hof1 and Rvs167 are not orthologues, they both share an analogous structure, with an F-BAR or BAR domain at the amino terminus, capable of inducing membrane curvature, and SH3 domains at the carboxyl terminus that bind to specific proline-rich targets. The SH3 domain of Rvs167 becomes essential for assembly of the actomyosin ring in cells lacking Hof1, suggesting that it helps to recruit a regulator of the actin cytoskeleton. This new function of Rvs167 appears to be independent of its known role as a regulator of the Arp2/3 actin nucleator, as actin ring assembly is not abolished by the simultaneous inactivation of Hof1 and Arp2/3. Instead we find that recruitment to the bud-neck of the Iqg1 actin regulator is defective in cells lacking Hof1 and Rvs167, though future studies will be needed to determine if this reflects a direct interaction between these factors. The redundant role of Hof1 in actin ring assembly suggests that the mechanism of actin ring assembly has been conserved to a greater extent across evolution than anticipated previously.},

author = {Nkosi, Pedro Junior and Targosz, Bianca-Sabrina and Labib, Karim and Sanchez-Diaz, Alberto},

doi = {10.1371/journal.pone.0057846},

editor = {Aspenstrom, Pontus},

file = {:Users/deepikaa/Library/Application Support/Mendeley Desktop/Downloaded/Nkosi et al. - 2013 - Hof1 and Rvs167 Have Redundant Roles in Actomyosin Ring Function during Cytokinesis in Budding Yeast.pdf:pdf},

issn = {1932-6203},

journal = {PLoS ONE},

month = {feb},

pages = {e57846},

publisher = {Public Library of Science},

title = {{Hof1 and Rvs167 Have Redundant Roles in Actomyosin Ring Function during Cytokinesis in Budding Yeast}},

url = {http://dx.plos.org/10.1371/journal.pone.0057846},

volume = {8},

year = {2013}

}

@article{Kishimoto2011,

abstract = {During endocytic vesicle formation, distinct subdomains along the membrane invagination are specified by different proteins, which bend the membrane and drive scission. Bin-Amphiphysin-Rvs (BAR) and Fer-CIP4 homology-BAR (F-BAR) proteins can induce membrane curvature and have been suggested to facilitate membrane invagination and scission. Two F-BAR proteins, Syp1 and Bzz1, are found at budding yeast endocytic sites. Syp1 arrives early but departs from the endocytic site before formation of deep membrane invaginations and scission. Using genetic, spatiotemporal, and ultrastructural analyses, we demonstrate that Bzz1, the heterodimeric BAR domain protein Rvs161/167, actin polymerization, and the lipid phosphatase Sjl2 cooperate, each through a distinct mechanism, to induce membrane scission in yeast. Additionally, actin assembly and Rvs161/167 cooperate to drive formation of deep invaginations. Finally, we find that Bzz1, acting at the invagination base, stabilizes endocytic sites and functions with Rvs161/167, localized along the tubule, to achieve proper endocytic membrane geometry necessary for efficient scission. Together, our results reveal that dynamic interplay between a lipid phosphatase, actin assembly, and membrane-sculpting proteins leads to proper membrane shaping, tubule stabilization, and scission.},

author = {Kishimoto, Takuma and Sun, Yidi and Buser, Christopher and Liu, Jian and Michelot, Alphee and Drubin, David G.},

doi = {10.1073/pnas.1113413108},

file = {:Users/deepikaa/Library/Application Support/Mendeley Desktop/Downloaded/JSVDN3IS/E979.full.pdf:pdf},

issn = {0027-8424},

journal = {Proceedings of the National Academy of Sciences of the United States of America},

month = {nov},

pages = {E979--E988},

title = {{Determinants of endocytic membrane geometry, stability, and scission}},

url = {http://www.ncbi.nlm.nih.gov/pmc/articles/PMC3207701/ http://www.ncbi.nlm.nih.gov/pmc/articles/PMC3207701/pdf/pnas.201113413.pdf},

volume = {108},

year = {2011}

}

@article{Shupliakov1997,

abstract = {The proline-rich COOH-terminal region of dynamin binds various Src homology 3 (SH3) domain-containing proteins, but the physiological role of these interactions is unknown. In living nerve terminals, the function of the interaction with SH3 domains was examined. Amphiphysin contains an SH3 domain and is a major dynamin binding partner at the synapse. Microinjection of amphiphysin's SH3 domain or of a dynamin peptide containing the SH3 binding site inhibited synaptic vesicle endocytosis at the stage of invaginated clathrin-coated pits, which resulted in an activity-dependent distortion of the synaptic architecture and a depression of transmitter release. These findings demonstrate that SH3-mediated interactions are required for dynamin function and support an essential role of clathrin-mediated endocytosis in synaptic vesicle recycling.},

author = {Shupliakov, O and L{\"{o}}w, P and Grabs, D and Gad, H and Chen, H and David, C and Takei, K and {De Camilli}, P and Brodin, L},

issn = {0036-8075},

journal = {Science},

month = {apr},

pages = {259--63},

pmid = {9092476},

title = {{Synaptic vesicle endocytosis impaired by disruption of dynamin-SH3 domain interactions.}},

url = {http://www.ncbi.nlm.nih.gov/pubmed/9092476},

volume = {276},

year = {1997}

}

@article{Boucrot2012,

abstract = {Shallow hydrophobic insertions and crescent-shaped BAR scaffolds promote membrane curvature. Here, we investigate membrane fission by shallow hydrophobic insertions quantitatively and mechanistically. We provide evidence that membrane insertion of the ENTH domain of epsin leads to liposome vesiculation, and that epsin is required for clathrin-coated vesicle budding in cells. We also show that BAR-domain scaffolds from endophilin, amphiphysin, GRAF, and $\beta$2-centaurin limit membrane fission driven by hydrophobic insertions. A quantitative assay for vesiculation reveals an antagonistic relationship between amphipathic helices and scaffolds of N-BAR domains in fission. The extent of vesiculation by these proteins and vesicle size depend on the number and length of amphipathic helices per BAR domain, in accord with theoretical considerations. This fission mechanism gives a new framework for understanding membrane scission in the absence of mechanoenzymes such as dynamin and suggests how Arf and Sar proteins work in vesicle scission., Shallow hydrophobic deformations of a membrane by amphipathic helices not only cause changes in curvature but also drive complete fission. This reaction is limited by BAR domains that are often found in the same proteins with amphipathic helices, suggesting that in vivo the propensity for a membrane to vesiculate can be regulated by a combinatory effect associated with protein domains.},

author = {Boucrot, Emmanuel and Pick, Adi and Camdere, Gamze and Liska, Nicole and Evergren, Emma and McMahon, Harvey T. and Kozlov, Michael M.},

doi = {10.1016/j.cell.2012.01.047},

issn = {0092-8674},

journal = {Cell},

keywords = {Folder - bar proteins},

mendeley-tags = {Folder - bar proteins},

month = {mar},

pages = {124--136},

title = {{Membrane Fission Is Promoted by Insertion of Amphipathic Helices and Is Restricted by Crescent BAR Domains}},

url = {http://www.ncbi.nlm.nih.gov/pmc/articles/PMC3465558/},

volume = {149},

year = {2012}

}

@article{Wang2011,

abstract = {The yeast dynamin-related GTPase Vps1 has been implicated in a range of cellular functions including vacuolar protein sorting, protein trafficking, organization of peroxisome and endocytosis.1,2 Vps1 is present at endocytic sites and may be directly involved in endocytic vesicle invagination through its membrane-tubulating activity. Here, evidence supporting the functional link between Vps1 and the yeast amphiphysin Rvs167 in vesicle invagination is discussed. Though the disassembly of endocytic factors from pinched-off endocytic vesicles appears to be tightly regulated in a spatiotemporal manner, we are far from having complete understanding of the underlying mechanism. In this study, we provide evidence that Vps1 plays a role in the uncoating of endocytic proteins from post-internalized vesicles, based on the observation of a quick disassembly of two endocytic coat proteins Ent1 and Ent2 in cells lacking Vps1.},

author = {Wang, Daobing and Sletto, Jeff and Tenay, Brandon and Kim, Kyoungtae},

doi = {10.4161/cib.4.2.14257},

issn = {1942-0889},

journal = {Communicative {\&} integrative biology},

keywords = {Folder - dynamin},

language = {eng},

mendeley-tags = {Folder - dynamin},

month = {mar},

pages = {178--181},

title = {{Yeast dynamin implicated in endocytic scission and the disassembly of endocytic components}},

url = {http://www.ncbi.nlm.nih.gov/pubmed/21655433},

volume = {4},

year = {2011}

}

@article{Sweitzer1998,

abstract = {The dynamin family of GTPases is essential for receptor-mediated endocytosis and synaptic vesicle recycling, and it has recently been shown to play a role in vesicle formation from the trans–Golgi network. Dynamin is believed to assemble around the necks of clathrin-coated pits and assist in pinching vesicles from the plasma membrane. This role would make dynamin unique among GTPases in its ability to act as a mechanochemical enzyme. Data presented here demonstrate that purified recombinant dynamin binds to a lipid bilayer in a regular pattern to form helical tubes that constrict and vesiculate upon GTP addition. This suggests that dynamin alone is sufficient for the formation of constricted necks of coated pits and supports the hypothesis that dynamin is the force-generating molecule responsible for membrane fission.},

author = {Sweitzer, Sharon M and Hinshaw, Jenny E},

doi = {10.1016/S0092-8674(00)81207-6},

file = {:Users/deepikaa/Library/Application Support/Mendeley Desktop/Downloaded/Sweitzer, Hinshaw - 1998 - Dynamin Undergoes a GTP-Dependent Conformational Change Causing Vesiculation.pdf:pdf},

issn = {0092-8674},

journal = {Cell},

month = {jun},

pages = {1021--1029},

publisher = {Cell Press},

title = {{Dynamin Undergoes a GTP-Dependent Conformational Change Causing Vesiculation}},

url = {https://www.sciencedirect.com/science/article/pii/S0092867400812076?via{\%}3Dihub},

volume = {93},

year = {1998}

}

@article{Kukulski2016c,

abstract = {{\textless}p{\textgreater}In a previous paper (Picco et al., 2015), the dynamic architecture of the protein machinery during clathrin-mediated endocytosis was visualized using a new live imaging and particle tracking method. Here, by combining this approach with correlative light and electron microscopy, we address the role of clathrin in this process. During endocytosis, clathrin forms a cage-like coat around the membrane and associated protein components. There is growing evidence that clathrin does not determine the membrane morphology of the invagination but rather modulates the progression of endocytosis. We investigate how the deletion of clathrin heavy chain impairs the dynamics and the morphology of the endocytic membrane in budding yeast. Our results show that clathrin is not required for elongating or shaping the endocytic membrane invagination. Instead, we find that clathrin contributes to the regularity of vesicle scission and thereby to controlling vesicle size.{\textless}/p{\textgreater}},

author = {Kukulski, Wanda and Picco, Andrea and Specht, Tanja and Briggs, John AG and Kaksonen, Marko},

doi = {10.7554/eLife.16036},

file = {:Users/deepikaa/Library/Application Support/Mendeley Desktop/Downloaded/Kukulski et al. - 2016 - Clathrin modulates vesicle scission, but not invagination shape, in yeast endocytosis.pdf:pdf},

issn = {2050-084X},

journal = {eLife},

keywords = {clathrin,correlative microscopy,endocytosis,live-imaging},

month = {jun},

pages = {e16036},

publisher = {eLife Sciences Publications Limited},

title = {{Clathrin modulates vesicle scission, but not invagination shape, in yeast endocytosis}},

url = {https://elifesciences.org/articles/16036},

volume = {5},

year = {2016}

}

@article{Sivadon1997,

abstract = {Mutations in RVS161 and RVS167 yeast genes induce identical phenotypes associated to actin cytoskeleton disorders. The whole Rvs161 protein is similar to the amino-terminal part of Rvs167p, thus defining a RVS domain. In addition to this domain, Rvs167p contains a central glycine-proline-alanine rich domain and a SH3 domain. To assess the function of these different domains we have expressed recombinant Rvs proteins in rvs mutant strains. Phenotype analysis has shown that the RVS and SH3 domains are necessary for phenotypical complementation, whereas the GPA domain is not. Moreover, we have demonstrated that the RVS domains from Rvs161p and Rvs167p have distinct roles, and that the SH3 domain needs the specific RVS domain of Rvs167p to function. These results suggest that Rvs161p and Rvs167p play distinct roles, while acting together in a common function.},

author = {Sivadon, P and Crouzet, M and Aigle, M},

issn = {0014-5793},

journal = {FEBS letters},

keywords = {Actins,Binding Sites,Cytoskeletal Proteins,Folder - bar proteins,Fungal Proteins,Microfilament Proteins,Phenotype,Recombinant Fusion Proteins,Recombinant Proteins,Saccharomyces cerevisiae,Saccharomyces cerevisiae Proteins,cytoskeleton},

language = {eng},

mendeley-tags = {Actins,Binding Sites,Cytoskeletal Proteins,Folder - bar proteins,Fungal Proteins,Microfilament Proteins,Phenotype,Recombinant Fusion Proteins,Recombinant Proteins,Saccharomyces cerevisiae,Saccharomyces cerevisiae Proteins,cytoskeleton},

month = {nov},

pages = {21--27},

title = {{Functional assessment of the yeast Rvs161 and Rvs167 protein domains}},

url = {http://www.ncbi.nlm.nih.gov/pubmed/9395067},

volume = {417},

year = {1997}

}

@article{Liu2009,

abstract = {An integrated theoretical model reveals how the chemical and the mechanical aspects of endocytosis are coordinated coherently in yeast cells, driving progression through the endocytic pathway and ensuring efficient vesicle scission in vivo.},

author = {Liu, Jian and Sun, Yidi and Drubin, David G. and Oster, George F.},

doi = {10.1371/journal.pbio.1000204},

file = {:Users/deepikaa/Library/Application Support/Mendeley Desktop/Downloaded/Liu et al. - 2009 - The Mechanochemistry of Endocytosis.pdf:pdf;:Users/deepikaa/Library/Application Support/Mendeley Desktop/Downloaded/82SAF7KM/infodoi10.1371journal.pbio.html:html},

journal = {PLoS Biol},

keywords = {BAR domain protein link,Folder - other proteins,Synaptotagmins},

mendeley-tags = {BAR domain protein link,Folder - other proteins,Synaptotagmins},

month = {sep},

pages = {e1000204},

title = {{The Mechanochemistry of Endocytosis}},

url = {http://dx.doi.org/10.1371/journal.pbio.1000204 http://www.plosbiology.org/article/fetchObject.action?uri=info{\%}3Adoi{\%}2F10.1371{\%}2Fjournal.pbio.1000204{\&}representation=PDF http://www.plosbiology.org/article/info{\%}3Adoi{\%}2F10.1371{\%}2Fjournal.pbio.1000204},

volume = {7},

year = {2009}

}

@article{Yu2004,

abstract = {Recent studies have suggested that the function of the large GTPase dynamin in endocytosis in mammalian cells may comprise a modulation of actin cytoskeleton. The role of dynamin in actin cytoskeleton organization in the yeast Saccharomyces cerevisiae has remained undefined. In this report, we found that one of the yeast dynamin-related proteins, Vps1p, is required for normal actin cytoskeleton organization. At both permissive and non-permissive temperatures, the vps1 mutants exhibited various degrees of phenotypes commonly associated with actin cytoskeleton defects: depolarized and aggregated actin structures, hypersensitivity to the actin cytoskeleton toxin latrunculin-A, randomized bud site selection and chitin deposition, and impaired efficiency in the internalization of membrane receptors. Over-expression of the GTPase mutants of vps1 also led to actin abnormalities. Consistent with these actin-related defects, Vps1p was found to interact physically, and partially co-localize, with the actin-regulatory protein Sla1p. The normal cellular localization of Sla1p required Vps1p and could be altered by over-expression of a region of Vps1p that was involved in the interaction with Sla1p. The same region also promoted mis-sorting of the vacuolar protein carboxypeptidase Y upon over-expression. These findings suggest that the functions of the dynamin-related protein Vps1p in actin cytoskeleton dynamics and vacuolar protein sorting are probably related to each other.},

author = {Yu, Xianwen and Cai, Mingjie},

doi = {10.1242/jcs.01239},

file = {:Users/deepikaa/Library/Application Support/Mendeley Desktop/Downloaded/Yu et al. - 2004 - The yeast dynamin-related GTPase Vps1p functions in the organization of the actin cytoskeleton via interaction with S.pdf:pdf;:Users/deepikaa/Library/Application Support/Mendeley Desktop/Downloaded/JUTAPPXB/3839.html:html},

issn = {0021-9533, 1477-9137},

journal = {Journal of Cell Science},

keywords = {Folder - now,GTPase,Sla1p,Vps1p,actin,dynamin},

language = {en},

mendeley-tags = {Folder - now,GTPase,Sla1p,Vps1p,actin,dynamin},

month = {aug},

pages = {3839--3853},

title = {{The yeast dynamin-related GTPase Vps1p functions in the organization of the actin cytoskeleton via interaction with Sla1p}},

url = {http://jcs.biologists.org/content/117/17/3839 http://jcs.biologists.org/content/joces/117/17/3839.full.pdf http://www.ncbi.nlm.nih.gov/pubmed/15265985},

volume = {117},

year = {2004}

}

@article{Lila1997,

author = {Lila, T. and Drubin, D. G.},

doi = {10.1091/mbc.8.2.367},

file = {:Users/deepikaa/Library/Application Support/Mendeley Desktop/Downloaded/JPVQ7QK5/367.html:html},

issn = {1059-1524, 1939-4586},

journal = {Molecular Biology of the Cell},

keywords = {Folder - bar proteins},

language = {en},

mendeley-tags = {Folder - bar proteins},

month = {feb},

pages = {367--385},

title = {{Evidence for physical and functional interactions among two Saccharomyces cerevisiae SH3 domain proteins, an adenylyl cyclase-associated protein and the actin cytoskeleton.}},

url = {http://www.molbiolcell.org/content/8/2/367 http://www.molbiolcell.org/content/8/2/367.full.pdf http://www.molbiolcell.org/content/8/2/367.long http://www.ncbi.nlm.nih.gov/pubmed/9190214},

volume = {8},

year = {1997}

}

@article{Simunovic2017b,

abstract = {Graphical Abstract Highlights d BAR protein scaffolds form a lipid diffusion barrier on membrane nanotubes d Elongation force on tubes reveals scaffold-membrane friction d Local tension rises due to friction, leading to pore nucleation and tube scission d Microtubule-associated molecular motors pull and cut scaffolded tubes SUMMARY Membrane scission is essential for intracellular traf-ficking. While BAR domain proteins such as endo-philin have been reported in dynamin-independent scission of tubular membrane necks, the cutting mechanism has yet to be deciphered. Here, we combine a theoretical model, in vitro, and in vivo experiments revealing how protein scaffolds may cut tubular membranes. We demonstrate that the protein scaffold bound to the underlying tube creates a frictional barrier for lipid diffusion; tube elongation thus builds local membrane tension until the membrane undergoes scission through lysis. We call this mechanism friction-driven scission (FDS). In cells, motors pull tubes, particularly during endocy-tosis. Through reconstitution, we show that motors not only can pull out and extend protein-scaffolded tubes but also can cut them by FDS. FDS is generic, operating even in the absence of amphipathic helices in the BAR domain, and could in principle apply to any high-friction protein and membrane assembly.},

author = {Simunovic, Mijo and Manneville, Jean-Baptiste and Renard, Henri-Fran{\c{c}} Ois and Johannes, Ludger and Bassereau, Patricia and Callan, Andrew},

doi = {10.1016/j.cell.2017.05.047},

file = {:Users/deepikaa/Library/Application Support/Mendeley Desktop/Downloaded/Yoshida et al. - 2004 - The stimulatory action of amphiphysin on dynamin function is dependent on lipid bilayer curvature.pdf:pdf},

journal = {Cell},

keywords = {BAR domain,diffusion barrier,endocytosis,endophilin,friction-driven scission,in vitro reconstitution,membrane scission,membrane tube,molecular motors,scaffold},

pages = {172-184.e11},

publisher = {Elsevier Inc},

title = {{Friction Mediates Scission of Tubular Membranes Scaffolded by BAR Proteins}},

url = {http://dx.doi.org/10.1016/j.cell.2017.05.047},

volume = {170},

year = {2017}

}

@article{Geli2000,

abstract = {The yeast type I myosins (MYO3 and MYO5) are involved in endocytosis and in the polarization of the actin cytoskeleton. The tail of these proteins contains a Tail Homology 2 (TH2) domain that constitutes a putative actin-binding site. Because of the important mechanistic implications of a second ATP-independent actin-binding site, we analyzed its functional relevance in vivo. Even though the myosin tail interacts with actin, and this interaction seems functionally important, deletion of a major portion of the TH2 domain did not abolish interaction. In contrast, we found that the SH3 domain of Myo5p significantly contributes to this interaction, implicating other proteins. We found that Vrp1p, the yeast homolog of WIP [Wiskott-Aldrich syndrome protein (WASP)-interacting protein], seems necessary to sustain the Myo5p tail-F-actin interaction. Consistent with recent results implicating the yeast type I myosins in regulating actin polymerization in vivo, we demonstrate that the C-terminal domain of Myo5p is able to induce cytosol-dependent actin polymerization in vitro, and that this activity requires both an intact Myo5p SH3 domain and Vrp1p.},

author = {Geli, M.I. and Lombardi, R and Schmelzl, B and Riezman, H},

doi = {10.1093/emboj/19.16.4281},

issn = {14602075},

journal = {The EMBO Journal},

month = {aug},

pages = {4281--4291},

pmid = {10944111},

title = {{An intact SH3 domain is required for myosin I-induced actin polymerization}},

url = {http://www.ncbi.nlm.nih.gov/pubmed/10944111 http://www.pubmedcentral.nih.gov/articlerender.fcgi?artid=PMC302045 http://emboj.embopress.org/cgi/doi/10.1093/emboj/19.16.4281},

volume = {19},

year = {2000}

}

@article{Kaksonen2018,

abstract = {Cellular membranes are formed from a chemically diverse set of lipids present in various amounts and proportions. A high lipid diversity is universal in eukaryotes and is seen from the scale of a membrane leaflet to that of a whole organism, highlighting its importance and suggesting that membrane lipids fulfil many functions. Indeed, alterations of membrane lipid homeostasis are linked to various diseases. While many of their functions remain unknown, interdisciplinary approaches have begun to reveal novel functions of lipids and their interactions. We are beginning to understand why even small changes in lipid structures and in composition can have profound effects on crucial biological functions.},

author = {Kaksonen, Marko and Roux, Aurelien},

doi = {10.1038/nrm.2017.132},

issn = {14710080},

journal = {Nature Reviews Molecular Cell Biology},

month = {may},

pages = {313--326},

file = {:Users/deepikaa/Library/Application Support/Mendeley Desktop/Downloaded/Kaksonen, Roux - 2018 - Mechanisms of clathrin-mediated endocytosis.pdf:pdf},

title = {{Mechanisms of clathrin-mediated endocytosis}},

url = {www.nature.com/nrm},

volume = {19},

year = {2018}

}

@article{Colwill1999,

abstract = {Morphological changes during cell division in the yeast Saccharomyces cerevisiae are controlled by cell-cycle regulators. The Pcl-Pho85p kinase complex has been implicated in the regulation of the actin cytoskeleton at least in part through Rvs167p. Rvs167p consists of three domains called BAR, GPA, and SH3. Using a two-hybrid assay, we demonstrated that each region of Rvs167p participates in protein-protein interactions: the BAR domain bound the BAR domain of another Rvs167p protein and that of Rvs161p, the GPA region bound Pcl2p, and the SH3 domain bound Abp1p. We identified Rvs167p as a Las17p/Bee1p-interacting protein in a two-hybrid screen and showed that Las17p/Bee1p bound the SH3 domain of Rvs167p. We tested the extent to which the Rvs167p protein domains rescued phenotypes associated with deletion of RVS167: salt sensitivity, random budding, and endocytosis and sporulation defects. The BAR domain was sufficient for full or partial rescue of all rvs167 mutant phenotypes tested but not required for the sporulation defect for which the SH3 domain was also sufficient. Overexpression of Rvs167p inhibits cell growth. The BAR domain was essential for this inhibition and the SH3 domain had only a minor effect. Rvs167p may link the cell cycle regulator Pcl-Pho85p kinase and the actin cytoskeleton. We propose that Rvs167p is activated by phosphorylation in its GPA region by the Pcl-Pho85p kinase. Upon activation, Rvs167p enters a multiprotein complex, making critical contacts in its BAR domain and redundant or minor contacts with its SH3 domain.},

author = {Colwill, Karen and Field, Deborah and Moore, Lynda and Friesen, James and Andrews, Brenda},

file = {:Users/deepikaa/Library/Application Support/Mendeley Desktop/Downloaded/Colwill et al. - Unknown - In Vivo Analysis of the Domains of Yeast Rvs167p Suggests Rvs167p Function Is Mediated Through Multiple Pr(2).pdf:pdf;:Users/deepikaa/Library/Application Support/Mendeley Desktop/Downloaded/U4TDC4ZK/881.html:html},

issn = {0016-6731, 1943-2631},

journal = {Genetics},

keywords = {Folder - bar proteins},

language = {en},

mendeley-tags = {Folder - bar proteins},

month = {jul},

pages = {881--893},

title = {{In Vivo Analysis of the Domains of Yeast Rvs167p Suggests Rvs167p Function Is Mediated Through Multiple Protein Interactions}},

url = {http://www.genetics.org/content/152/3/881 http://www.genetics.org/content/152/3/881.full.pdf http://www.ncbi.nlm.nih.gov/pubmed/10388809},

volume = {152},

year = {1999}

}

@article{Vater1992,

abstract = {The product of the VPS1 gene, Vps1p, is required for the sorting of soluble vacuolar proteins in the yeast Saccharomyces cerevisiae. We demonstrate here that Vps1p, which contains a consensus tripartite motif for guanine nucleotide binding, is capable of binding and hydrolyzing GTP. Vps1p is a member of a subfamily of large GTP-binding proteins whose members include the vertebrate Mx proteins, the yeast MGM1 protein, the Drosophila melanogaster shibire protein, and dynamin, a bovine brain protein that bundles microtubules in vitro. Disruption of microtubules did not affect the fidelity or kinetics of vacuolar protein sorting, indicating that Vps1p function is not dependent on microtubules. Based on mutational analyses, we propose a two-domain model for Vps1p function. When VPS1 was treated with hydroxylamine, half of all mutations isolated were found to be dominant negative with respect to vacuolar protein sorting. All of the dominant-negative mutations analyzed further mapped to the amino-terminal half of Vps1p and gave rise to full-length protein products. In contrast, recessive mutations gave rise to truncated or unstable protein products. Two large deletion mutations in VPS1 were created to further investigate Vps1p function. A mutant form of Vps1p lacking the carboxy-terminal half of the protein retained the capacity to bind GTP and did not interfere with sorting in a wild-type background. A mutant form of Vps1p lacking the entire GTP-binding domain interfered with vacuolar protein sorting in wild-type cells. We suggest that the amino-terminal domain of Vps1p provides a GTP-binding and hydrolyzing activity required for vacuolar protein sorting, and the carboxy-terminal domain mediates Vps1p association with an as yet unidentified component of the sorting apparatus.},

author = {Vater, C A and Raymond, C K and Ekena, K and Howald-Stevenson, I and Stevens, T H},

issn = {0021-9525},

journal = {The Journal of cell biology},

month = {nov},

pages = {773--86},

pmid = {1429836},

publisher = {The Rockefeller University Press},

title = {{The VPS1 protein, a homolog of dynamin required for vacuolar protein sorting in Saccharomyces cerevisiae, is a GTPase with two functionally separable domains.}},

url = {http://www.ncbi.nlm.nih.gov/pubmed/1429836 http://www.pubmedcentral.nih.gov/articlerender.fcgi?artid=PMC2289700},

volume = {119},

year = {1992}

}

@article{Qualmann2011,

abstract = {Against the odds of membrane resistance, members of the BIN/Amphiphysin/Rvs (BAR) domain superfamily shape membranes and their activity is indispensable for a plethora of life functions. While crystal structures of different BAR dimers advanced our understanding of membrane shaping by scaffolding and hydrophobic insertion mechanisms considerably, especially life‐imaging techniques and loss‐of‐function studies of clathrin‐mediated endocytosis with its gradually increasing curvature show that the initial idea that solely BAR domain curvatures determine their functions is oversimplified. Diagonal placing, lateral lipid‐binding modes, additional lipid‐binding modules, tilde shapes and formation of macromolecular lattices with different modes of organisation and arrangement increase versatility. A picture emerges, in which BAR domain proteins create macromolecular platforms, that recruit and connect different binding partners and ensure the connection and coordination of the different events during the endocytic process, such as membrane invagination, coat formation, actin nucleation, vesicle size control, fission, detachment and uncoating, in time and space, and may thereby offer mechanistic explanations for how coordination, directionality and effectiveness of a complex process with several steps and key players can be achieved.},

author = {Qualmann, Britta and Koch, Dennis and Kessels, Michael Manfred},

doi = {10.1038/emboj.2011.266},

file = {:Users/deepikaa/Library/Application Support/Mendeley Desktop/Downloaded/Zhu et al. - 2007 - Structure of the APPL1 BAR-PH domain and characterization of its interaction with Rab5.pdf:pdf;:Users/deepikaa/Library/Application Support/Mendeley Desktop/Downloaded/9NCK3WJJ/3501.html:html},

issn = {0261-4189, 1460-2075},

journal = {The EMBO Journal},

keywords = {BAR domain superfamily proteins,BAR hypothesis,Folder - now,clathrin‐mediated endocytosis,curvature,vesicle formation},

language = {en},

mendeley-tags = {BAR domain superfamily proteins,BAR hypothesis,Folder - now,clathrin‐mediated endocytosis,curvature,vesicle formation},

month = {aug},

pages = {3501--3515},

shorttitle = {Let's go bananas},

title = {{Let's go bananas: revisiting the endocytic BAR code}},

url = {http://emboj.embopress.org/content/30/17/3501 http://emboj.embopress.org/content/embojnl/30/17/3501.full.pdf http://www.ncbi.nlm.nih.gov/pubmed/21878992},

volume = {30},

year = {2011}

}

@article{GoudGadila2017,

abstract = {The yeast dynamin Vps1 acts cooperatively with many proteins at diverse cellular locations for endocytosis, protein sorting, and membrane fusion and fission. It has been proposed that Vps1 is functionally linked to clathrin heavy chain 1 (Chc1), but the question of how, where, and when they function together remains unknown. Here we report that Vps1 arrives at the Golgi after clathrin, and that loss of Vps1 leads to a shift in the cellular localization of clathrin to the late endosome and vacuole, not vice versa. Our two-hybrid-based approach provides evidence that full-length Vps1 and its truncated versions bind to the C-terminal region of the Chc1. Cells lacking both Vps1 and Chc1 displayed more severe defects in carboxypeptidase Y (CPY) sorting at the Golgi than those in Vps1-deficient cells. Further, these Vps1 fragments became dominant-negative for CPY sorting upon overexpression. These results suggest that Vps1 binds to Chc1 and functions together at the Golgi for efficient Golgi-to-endosome membrane trafficking. In addition, we found that Vps1, without the aid of clathrin, plays a role in controlling the number and turnover of late Golgi.},

author = {{Goud Gadila}, Shiva Kumar and Williams, Michelle and Saimani, Uma and {Delgado Cruz}, Mariel and Makaraci, Pelin and Woodman, Sara and Short, John C.W. and McDermott, Hyoeun and Kim, Kyoungtae},

doi = {10.1016/J.EJCB.2017.02.004},

file = {:Users/deepikaa/Library/Application Support/Mendeley Desktop/Downloaded/Goud Gadila et al. - 2017 - Yeast dynamin Vps1 associates with clathrin to facilitate vesicular trafficking and controls Golgi homeostas.pdf:pdf},

issn = {0171-9335},

journal = {European Journal of Cell Biology},

month = {mar},

pages = {182--197},

publisher = {Urban {\&} Fischer},

title = {{Yeast dynamin Vps1 associates with clathrin to facilitate vesicular trafficking and controls Golgi homeostasis}},

url = {https://www.sciencedirect.com/science/article/pii/S0171933516302412},

volume = {96},

year = {2017}

}

@article{Bitsikas2014,

abstract = {{\textless}p{\textgreater}Several different endocytic pathways have been proposed to function in mammalian cells. Clathrin-coated pits are well defined, but the identity, mechanism and function of alternative pathways have been controversial. Here we apply universal chemical labelling of plasma membrane proteins to define all primary endocytic vesicles, and labelling of specific proteins with a reducible SNAP-tag substrate. These approaches provide high temporal resolution and stringent discrimination between surface-connected and intracellular membranes. We find that at least 95{\%} of the earliest detectable endocytic vesicles arise from clathrin-coated pits. GPI-anchored proteins, candidate cargoes for alternate pathways, are also found to enter the cell predominantly via coated pits. Experiments employing a mutated clathrin adaptor reveal distinct mechanisms for sorting into coated pits, and thereby explain differential effects on the uptake of transferrin and GPI-anchored proteins. These data call for a revision of models for the activity and diversity of endocytic pathways in mammalian cells.{\textless}/p{\textgreater}},

author = {Bitsikas, Vassilis and Corr{\^{e}}a, Ivan R and Nichols, Benjamin J},

doi = {10.7554/eLife.03970},

file = {:Users/deepikaa/Library/Application Support/Mendeley Desktop/Downloaded/Bitsikas, Corr{\^{e}}a, Nichols - 2014 - Clathrin-independent pathways do not contribute significantly to endocytic flux.pdf:pdf},

issn = {2050-084X},

journal = {eLife},

keywords = {clathrin,endocytosis,membrane transport,vesicle},

month = {sep},

pages = {e03970},

publisher = {eLife Sciences Publications Limited},

title = {{Clathrin-independent pathways do not contribute significantly to endocytic flux}},

url = {https://elifesciences.org/articles/03970},

volume = {3},

year = {2014}

}

@article{Youn2010,

abstract = {Using a structure–function analysis, we find that Rvs proteins are initially recruited to sites of endocytosis through their curvature-sensing and membrane-binding ability in a manner dependent on complex sphingolipids., BAR domains are protein modules that bind to membranes and promote membrane curvature. One type of BAR domain, the N-BAR domain, contains an additional N-terminal amphipathic helix, which contributes to membrane-binding and bending activities. The only known N-BAR-domain proteins in the budding yeast Saccharomyces cerevisiae, Rvs161 and Rvs167, are required for endocytosis. We have explored the mechanism of N-BAR-domain function in the endocytosis process using a combined biochemical and genetic approach. We show that the purified Rvs161–Rvs167 complex binds to liposomes in a curvature-independent manner and promotes tubule formation in vitro. Consistent with the known role of BAR domain polymerization in membrane bending, we found that Rvs167 BAR domains interact with each other at cortical actin patches in vivo. To characterize N-BAR-domain function in endocytosis, we constructed yeast strains harboring changes in conserved residues in the Rvs161 and Rvs167 N-BAR domains. In vivo analysis of the rvs endocytosis mutants suggests that Rvs proteins are initially recruited to sites of endocytosis through their membrane-binding ability. We show that inappropriate regulation of complex sphingolipid and phosphoinositide levels in the membrane can impinge on Rvs function, highlighting the relationship between membrane components and N-BAR-domain proteins in vivo.},

author = {Youn, Ji-Young and Friesen, Helena and Kishimoto, Takuma and Henne, William M. and Kurat, Christoph F. and Ye, Wei and Ceccarelli, Derek F. and Sicheri, Frank and Kohlwein, Sepp D. and McMahon, Harvey T. and Andrews, Brenda J.},

doi = {10.1091/mbc.E10-03-0181},

file = {:Users/deepikaa/Library/Application Support/Mendeley Desktop/Downloaded/ENEVHWB4/Youn et al. - 2010 - Dissecting BAR Domain Function in the Yeast Amphip.pdf:pdf},

issn = {1059-1524},

journal = {Molecular Biology of the Cell},

keywords = {Folder - bar proteins},

mendeley-tags = {Folder - bar proteins},

month = {sep},

pages = {3054--3069},

title = {{Dissecting BAR Domain Function in the Yeast Amphiphysins Rvs161 and Rvs167 during Endocytosis}},

url = {http://www.ncbi.nlm.nih.gov/pmc/articles/PMC2929998/ http://www.ncbi.nlm.nih.gov/pmc/articles/PMC2929998/pdf/zmk01710003054.pdf},

volume = {21},

year = {2010}

}

@article{Ferguson2007,

abstract = {Dynamin 1 is a neuron-specific guanosine triphosphatase thought to be critically required for the fission reaction of synaptic vesicle endocytosis. Unexpectedly, mice lacking dynamin 1 were able to form functional synapses, even though their postnatal viability was limited. However, during spontaneous network activity, branched, tubular plasma membrane invaginations accumulated, capped by clathrin-coated pits, in synapses of dynamin 1-knockout mice. Synaptic vesicle endocytosis was severely impaired during strong exogenous stimulation but resumed efficiently when the stimulus was terminated. Thus, dynamin 1-independent mechanisms can support limited synaptic vesicle endocytosis, but dynamin 1 is needed during high levels of neuronal activity.},

author = {Ferguson, Shawn M. and Brasnjo, Gabor and Hayashi, Mitsuko and W{\"{o}}lfel, Markus and Collesi, Chiara and Giovedi, Silvia and Raimondi, Andrea and Gong, Liang Wei and Ariel, Pablo and Paradise, Summer and O'Toole, Eileen and Flavell, Richard and Cremona, Ottavio and Miesenb{\"{o}}ck, Gero and Ryan, Timothy A. and {De Camilli}, Pietro},

doi = {10.1126/science.1140621},

file = {:Users/deepikaa/Library/Application Support/Mendeley Desktop/Downloaded/Ferguson et al. - 2007 - A selective activity-dependent requirement for dynamin 1 in synaptic vesicle endocytosis.pdf:pdf},

issn = {00368075},

journal = {Science},

keywords = {Action Potentials,Animals,Cell Membrane / ultrastructure,Clathrin-Coated Vesicles / metabolism,Clathrin-Coated Vesicles / ultrastructure,Dynamin I / genetics,Dynamin I / physiology*,Dynamin II,Dynamin III / physiology,Electric Stimulation,Electron,Endocytosis*,Excitatory Postsynaptic Potentials,Exocytosis,Extramural,Gabor Brasnjo,Inhibitory Postsynaptic Potentials,Knockout,MEDLINE,Mice,Microscopy,N.I.H.,NCBI,NIH,NLM,National Center for Biotechnology Information,National Institutes of Health,National Library of Medicine,Neurons / physiology*,Neurons / ultrastructure,Non-U.S. Gov't,Patch-Clamp Techniques,Pietro De Camilli,Presynaptic Terminals / physiology,Presynaptic Terminals / ultrastructure,PubMed Abstract,Research Support,Shawn M Ferguson,Synapses / physiology*,Synapses / ultrastructure,Synaptic Transmission,Synaptic Vesicles / physiology*,Synaptic Vesicles / ultrastructure,doi:10.1126/science.1140621,pmid:17463283},

month = {apr},

pages = {570--574},

publisher = {Science},

title = {{A selective activity-dependent requirement for dynamin 1 in synaptic vesicle endocytosis}},

url = {https://pubmed.ncbi.nlm.nih.gov/17463283/ https://pubmed.ncbi.nlm.nih.gov/17463283/?dopt=Abstract},

volume = {316},

year = {2007}

}

@article{Mim2012,

abstract = {Functioning as key players in cellular regulation of membrane curvature, BAR domain proteins bend bilayers and recruit interaction partners through poorly understood mechanisms. Using electron cryomicroscopy, we present reconstructions of full-length endophilin and its N-terminal N-BAR domain in their membrane-bound state. Endophilin lattices expose large areas of membrane surface and are held together by promiscuous interactions between endophilin's amphipathic N-terminal helices. Coarse-grained molecular dynamics simulations reveal that endophilin lattices are highly dynamic and that the N-terminal helices are required for formation of a stable and regular scaffold. Furthermore, endophilin accommodates different curvatures through a quantized addition or removal of endophilin dimers, which in some cases causes dimerization of endophilin's SH3 domains, suggesting that the spatial presentation of SH3 domains, rather than affinity, governs the recruitment of downstream interaction partners.},

author = {Mim, Carsten and Cui, Haosheng and Gawronski-Salerno, Joseph A and Frost, Adam and Lyman, Edward and Voth, Gregory A and Unger, Vinzenz M},

doi = {10.1016/j.cell.2012.01.048},

issn = {1097-4172},

journal = {Cell},

keywords = {Acyltransferases,Animals,Cell Membrane,Cryoelectron Microscopy,Folder - bar proteins,Models- Molecular,Nerve Tissue Proteins,Protein Interaction Domains and Motifs,Protein Structure- Tertiary,Rats,structural},

language = {eng},

mendeley-tags = {Acyltransferases,Animals,Cell Membrane,Cryoelectron Microscopy,Folder - bar proteins,Models- Molecular,Nerve Tissue Proteins,Protein Interaction Domains and Motifs,Protein Structure- Tertiary,Rats,structural},

month = {mar},

pages = {137--145},

title = {{Structural basis of membrane bending by the N-BAR protein endophilin}},

url = {http://www.ncbi.nlm.nih.gov/pubmed/22464326},

volume = {149},

year = {2012}

}

@article{Kukulski2012,

abstract = {Summary

Endocytosis, like many dynamic cellular processes, requires precise temporal and spatial orchestration of complex protein machinery to mediate membrane budding. To understand how this machinery works, we directly correlated fluorescence microscopy of key protein pairs with electron tomography. We systematically located 211 endocytic intermediates, assigned each to a specific time window in endocytosis, and reconstructed their ultrastructure in 3D. The resulting virtual ultrastructural movie defines the protein-mediated membrane shape changes during endocytosis in budding yeast. It reveals that clathrin is recruited to flat membranes and does not initiate curvature. Instead, membrane invagination begins upon actin network assembly followed by amphiphysin binding to parallel membrane segments, which promotes elongation of the invagination into a tubule. Scission occurs on average 9 s after initial bending when invaginations are ∼100 nm deep, releasing nonspherical vesicles with 6,400 nm2 mean surface area. Direct correlation of protein dynamics with ultrastructure provides a quantitative 4D resource.},

author = {Kukulski, Wanda and Schorb, Martin and Kaksonen, Marko and Briggs, John A. G.},

doi = {10.1016/j.cell.2012.05.046},

file = {:Users/deepikaa/Library/Application Support/Mendeley Desktop/Downloaded/T96TN2K5/Kukulski et al. - 2012 - Plasma Membrane Reshaping during Endocytosis Is Re.pdf:pdf;:Users/deepikaa/Library/Application Support/Mendeley Desktop/Downloaded/XM4GN7IB/S0092867412007842.html:html},

issn = {0092-8674},

journal = {Cell},

keywords = {Folder - bar proteins,Folder - other proteins,electron tomography,s. cervisiae},

mendeley-tags = {Folder - bar proteins,Folder - other proteins,electron tomography,s. cervisiae},

month = {aug},

pages = {508--520},

title = {{Plasma Membrane Reshaping during Endocytosis Is Revealed by Time-Resolved Electron Tomography}},

url = {http://www.sciencedirect.com/science/article/pii/S0092867412007842 http://www.sciencedirect.com/science/article/pii/S0092867412007842/pdfft?md5=4f0c76f5c0723dd7745907c828670fe7{\&}pid=1-s2.0-S0092867412007842-main.pdf},

volume = {150},

year = {2012}

}

@article{Picco2015,

abstract = {Clathrin-mediated endocytosis is an essential process that forms vesicles from the plasma membrane. Although most of the protein components of the endocytic protein machinery have been thoroughly characterized, their organization at the endocytic site is poorly understood. We developed a fluorescence microscopy method to track the average positions of yeast endocytic proteins in relation to each other with a time precision below 1 s and with a spatial precision of {\~{}}10 nm. With these data, integrated with shapes of endocytic membrane intermediates and with superresolution imaging, we could visualize the dynamic architecture of the endocytic machinery. We showed how different coat proteins are distributed within the coat structure and how the assembly dynamics of N-BAR proteins relate to membrane shape changes. Moreover, we found that the region of actin polymerization is located at the base of the endocytic invagination, with the growing ends of filaments pointing toward the plasma membrane.To Top

Clathrin-mediated endocytosis is an essential process that forms vesicles from the plasma membrane. Although most of the protein components of the endocytic protein machinery have been thoroughly characterized, their organization at the endocytic site is poorly understood. We developed a fluorescence microscopy method to track the average positions of yeast endocytic proteins in relation to each other with a time precision below 1 s and with a spatial precision of {\~{}}10 nm. With these data, integrated with shapes of endocytic membrane intermediates and with superresolution imaging, we could visualize the dynamic architecture of the endocytic machinery. We showed how different coat proteins are distributed within the coat structure and how the assembly dynamics of N-BAR proteins relate to membrane shape changes. Moreover, we found that the region of actin polymerization is located at the base of the endocytic invagination, with the growing ends of filaments pointing toward the plasma membrane.},

author = {Picco, Andrea and Mund, Markus and Ries, Jonas and N{\'{e}}d{\'{e}}lec, Fran{\c{c}}ois and Kaksonen, Marko},

doi = {10.7554/eLife.04535},

file = {:Users/deepikaa/Library/Application Support/Mendeley Desktop/Downloaded/TARJ58Q7/Picco et al. - 2015 - Visualizing the functional architecture of the end.pdf:pdf;:Users/deepikaa/Library/Application Support/Mendeley Desktop/Downloaded/6UJIEBIU/eLife.html:html},

issn = {2050-084X},

journal = {eLife},

keywords = {Folder - other proteins},

language = {en},

mendeley-tags = {Folder - other proteins},

month = {feb},

pages = {e04535},

volume = {4},

title = {{Visualizing the functional architecture of the endocytic machinery}},

url = {http://elifesciences.org/content/early/2015/02/12/eLife.04535 http://elifesciences.org/content/early/2015/02/12/eLife.04535.full.pdf},

year = {2015}

}

@article{Peters2004,

abstract = {Membrane fusion and fission are antagonistic reactions controlled by different proteins. Dynamins promote membrane fission by GTP-driven changes of conformation and polymerization state, while SNAREs fuse membranes by forming complexes between t- and v-SNAREs from apposed vesicles. Here, we describe a role of the dynamin-like GTPase Vps1p in fusion of yeast vacuoles. Vps1p forms polymers that couple several t-SNAREs together. At the onset of fusion, the SNARE-activating ATPase Sec18p/NSF and the t-SNARE depolymerize Vps1p and release it from the membrane. This activity is independent of the SNARE coactivator Sec17p/alpha-SNAP and of the v-SNARE. Vps1p release liberates the t-SNAREs for initiating fusion and at the same time disrupts fission activity. We propose that reciprocal control between fusion and fission components exists, which may prevent futile cycles of fission and fusion.},

author = {Peters, Christopher and Baars, Tonie L and B{\"{u}}hler, Susanne and Mayer, Andreas},

doi = {10.1016/j.cell.2004.11.023},

file = {:Users/deepikaa/Library/Application Support/Mendeley Desktop/Downloaded/Peters et al. - 2004 - Mutual control of membrane fission and fusion proteins.pdf:pdf},

issn = {0092-8674},

journal = {Cell},

month = {nov},

pages = {667--78},

pmid = {15550248},

publisher = {Elsevier},

title = {{Mutual control of membrane fission and fusion proteins.}},

url = {http://www.ncbi.nlm.nih.gov/pubmed/15550248},

volume = {119},

year = {2004}

}

@article{Galli2017,

abstract = {{\textless}p{\textgreater}Dynamin is a large GTPase that forms a helical collar at the neck of endocytic pits, and catalyzes membrane fission (Schmid and Frolov, 2011; Ferguson and De Camilli, 2012). Dynamin fission reaction is strictly dependent on GTP hydrolysis, but how fission is mediated is still debated (Antonny et al., 2016): GTP energy could be spent in membrane constriction required for fission, or in disassembly of the dynamin polymer to trigger fission. To follow dynamin GTP hydrolysis at endocytic pits, we generated a conformation-specific nanobody called dynab, that binds preferentially to the GTP hydrolytic state of dynamin-1. Dynab allowed us to follow the GTPase activity of dynamin-1 in real-time. We show that in fibroblasts, dynamin GTP hydrolysis occurs as stochastic bursts, which are randomly distributed relatively to the peak of dynamin assembly. Thus, dynamin disassembly is not coupled to GTPase activity, supporting that the GTP energy is primarily spent in constriction.{\textless}/p{\textgreater}},

author = {Galli, Valentina and Sebastian, Rafael and Moutel, Sandrine and Ecard, Jason and Perez, Franck and Roux, Aur{\'{e}}lien},

doi = {10.7554/eLife.25197},

file = {:Users/deepikaa/Library/Application Support/Mendeley Desktop/Downloaded/Galli et al. - 2017 - Uncoupling of dynamin polymerization and GTPase activity revealed by the conformation-specific nanobody dynab.pdf:pdf},

issn = {2050-084X},

journal = {eLife},

keywords = {conformational-specific nanobody,dynamin,endocytosis,enzyme},

month = {oct},

pages = {e25197},

publisher = {eLife Sciences Publications Limited},

title = {{Uncoupling of dynamin polymerization and GTPase activity revealed by the conformation-specific nanobody dynab}},

url = {https://elifesciences.org/articles/25197},

volume = {6},

year = {2017}

}

@article{Bensen2000,

abstract = {Clathrin is involved in selective protein transport at the Golgi apparatus and the plasma membrane. To further understand the molecular mechanisms underlying clathrin-mediated protein transport pathways, we initiated a genetic screen for mutations that display synthetic growth defects when combined with a temperature-sensitive allele of the clathrin heavy chain gene (chc1-521) in Saccharomyces cerevisiae. Mutations, when present in cells with wild-type clathrin, were analyzed for effects on mating pheromone alpha-factor precursor maturation and sorting of the vacuolar protein carboxypeptidase Y as measures of protein sorting at the yeast trans-Golgi network (TGN) compartment. By these criteria, two classes of mutants were obtained, those with and those without defects in protein sorting at the TGN. One mutant with unaltered protein sorting at the TGN contains a mutation in PTC1, a type 2c serine/threonine phosphatase with widespread influences. The collection of mutants displaying TGN sorting defects includes members with mutations in previously identified vacuolar protein sorting genes (VPS), including the dynamin family member VPS1. Striking genetic interactions were observed by combining temperature-sensitive alleles of CHC1 and VPS1, supporting the model that Vps1p is involved in clathrin-mediated vesicle formation at the TGN. Also in the spectrum of mutants with TGN sorting defects are isolates with mutations in the following: RIC1, encoding a product originally proposed to participate in ribosome biogenesis; LUV1, encoding a product potentially involved in vacuole and microtubule organization; and INP53, encoding a synaptojanin-like inositol polyphosphate 5-phosphatase. Disruption of INP53, but not the related INP51 and INP52 genes, resulted in alpha-factor maturation defects and exacerbated alpha-factor maturation defects when combined with chc1-521. Our findings implicate a wide variety of proteins in clathrin-dependent processes and provide evidence for the selective involvement of Inp53p in clathrin-mediated protein sorting at the TGN.},

author = {Bensen, E S and Costaguta, G and Payne, G S},

issn = {0016-6731},

journal = {Genetics},

month = {jan},

pages = {83--97},

pmid = {10628971},

title = {{Synthetic genetic interactions with temperature-sensitive clathrin in Saccharomyces cerevisiae. Roles for synaptojanin-like Inp53p and dynamin-related Vps1p in clathrin-dependent protein sorting at the trans-Golgi network.}},

url = {http://www.ncbi.nlm.nih.gov/pubmed/10628971 http://www.pubmedcentral.nih.gov/articlerender.fcgi?artid=PMC1460916},

volume = {154},

year = {2000}

}

@article{Boeke2014b,

author = {Boeke, D. and Trautmann, S. and Meurer, M. and Wachsmuth, M. and Godlee, C. and Knop, M. and Kaksonen, M.},

doi = {10.15252/msb.20145422},

issn = {1744-4292},

journal = {Molecular Systems Biology},

keywords = {Folder - review},

language = {en},

mendeley-tags = {Folder - review},

month = {nov},

pages = {756--756},

title = {{Quantification of cytosolic interactions identifies Ede1 oligomers as key organizers of endocytosis}},

url = {http://msb.embopress.org/cgi/doi/10.15252/msb.20145422},

volume = {10},

year = {2014}

}

@article{Sun2006,

abstract = {Actin polymerization essential for endocytic internalization in budding yeast is controlled by four nucleation promoting factors (NPFs) that each exhibits a unique dynamic behavior at endocytic sites. How each NPF functions and is regulated to restrict actin assembly to late stages of endocytic internalization is not known. Quantitative analysis of NPF biochemical activities, and genetic analysis of recruitment and regulatory mechanisms, defined a linear pathway in which protein composition changes at endocytic sites control actin assembly and function. We show that yeast WASP initiates actin assembly at endocytic sites and that this assembly and the recruitment of a yeast WIP-like protein by WASP recruit a type I myosin with both NPF and motor activities. Importantly, type I myosin motor and NPF activities are separable, and both contribute to endocytic coat inward movement, which likely represents membrane invagination. These results reveal a mechanism in which actin nucleation and myosin motor activity cooperate to promote endocytic internalization.},

author = {Sun, Yidi and Martin, Adam C. and Drubin, David G.},

doi = {10.1016/j.devcel.2006.05.008},

issn = {15345807},

journal = {Developmental Cell},

month = {jul},

pages = {33--46},

pmid = {16824951},

title = {{Endocytic Internalization in Budding Yeast Requires Coordinated Actin Nucleation and Myosin Motor Activity}},

url = {http://www.ncbi.nlm.nih.gov/pubmed/16824951 http://linkinghub.elsevier.com/retrieve/pii/S1534580706002462},

volume = {11},

year = {2006}

}

@article{Cestra1999,

abstract = {The proline-rich domain of synaptojanin 1, a synaptic protein with phosphatidylinositol phosphatase activity, binds to amphiphysin and to a family of recently discovered proteins known as the SH3p4/8/13, the SH3-GL, or the endophilin family. These interactions are mediated by SH3 domains and are believed to play a regulatory role in synaptic vesicle recycling. We have precisely mapped the target peptides on human synaptojanin that are recognized by the SH3 domains of endophilins and amphiphysin and proven that they are distinct. By a combination of different approaches, selection of phage displayed peptide libraries, substitution analyses of peptides synthesized on cellulose membranes, and a peptide scan spanning a 252-residue long synaptojanin fragment, we have concluded that amphiphysin binds to two sites, PIRPSR and PTIPPR, whereas endophilin has a distinct preferred binding site, PKRPPPPR. The comparison of the results obtained by phage display and substitution analysis permitted the identification of proline and arginine at positions 4 and 6 in the PIRPSR and PTIPPR target sequence as the major determinants of the recognition specificity mediated by the SH3 domain of amphiphysin 1. More complex is the structural rationalization of the preferred endophilin ligands where SH3 binding cannot be easily interpreted in the framework of the "classical" type I or type II SH3 binding models. Our results suggest that the binding repertoire of SH3 domains may be more complex than originally predicted.},

author = {Cestra, G and Castagnoli, L and Dente, L and Minenkova, O and Petrelli, A and Migone, N and Hoffm{\"{u}}ller, U and Schneider-Mergener, J and Cesareni, G},

doi = {10.1074/JBC.274.45.32001},

file = {:Users/deepikaa/Library/Application Support/Mendeley Desktop/Downloaded/Cestra et al. - 1999 - The SH3 domains of endophilin and amphiphysin bind to the proline-rich region of synaptojanin 1 at distinct si(3).pdf:pdf},

issn = {0021-9258},

journal = {The Journal of biological chemistry},

month = {nov},

pages = {32001--7},

pmid = {10542231},

publisher = {American Society for Biochemistry and Molecular Biology},

title = {{The SH3 domains of endophilin and amphiphysin bind to the proline-rich region of synaptojanin 1 at distinct sites that display an unconventional binding specificity.}},

url = {http://www.ncbi.nlm.nih.gov/pubmed/10542231},

volume = {274},

year = {1999}

}

@article{Friesen2006,

abstract = {We have used comprehensive synthetic lethal screens and biochemical assays to examine the biological role of the yeast amphiphysin homologues Rvs161p and Rvs167p, two proteins that play a role in regulation of the actin cytoskeleton, endocytosis, and sporulation. We found that unlike some forms of amphiphysin, Rvs161p-Rvs167p acts as an obligate heterodimer during vegetative growth and neither Rvs161p nor Rvs167p forms a homodimer in vivo. RVS161 and RVS167 have an identical set of 49 synthetic lethal interactions, revealing functions for the Rvs proteins in cell polarity, cell wall synthesis, and vesicle trafficking as well as a shared role in mating. Consistent with these roles, we show that the Rvs167p-Rvs161p heterodimer, like its amphiphysin homologues, can bind to phospholipid membranes in vitro, suggesting a role in vesicle formation and/or fusion. Our genetic screens also reveal that the interaction between Abp1p and the Rvs167p Src homology 3 (SH3) domain may be important under certain conditions, providing the first genetic evidence for a role for the SH3 domain of Rvs167p. Our studies implicate heterodimerization of amphiphysin family proteins in various functions related to cell polarity, cell integrity, and vesicle trafficking during vegetative growth and the mating response.},

author = {Friesen, Helena and Humphries, Christine and Ho, Yuen and Schub, Oliver and Colwill, Karen and Andrews, Brenda},

doi = {10.1091/mbc.E05-06-0476},

file = {:Users/deepikaa/Library/Application Support/Mendeley Desktop/Downloaded/Friesen et al. - 2006 - Characterization of the yeast amphiphysins Rvs161p and Rvs167p reveals roles for the Rvs heterodimer in vivo.pdf:pdf},

issn = {1059-1524},

journal = {Molecular Biology of the Cell},

keywords = {Folder - bar proteins},

mendeley-tags = {Folder - bar proteins},

month = {mar},

pages = {1306--1321},

title = {{Characterization of the Yeast Amphiphysins Rvs161p and Rvs167p Reveals Roles for the Rvs Heterodimer In Vivo}},

url = {http://www.ncbi.nlm.nih.gov/pmc/articles/PMC1382319/ http://www.ncbi.nlm.nih.gov/pmc/articles/PMC1382319/pdf/1306.pdf},

volume = {17},

year = {2006}

}

@article{Rothman1986,

abstract = {We have devised a genetic selection for mutant yeast cells that fail to properly deliver the vacuolar glycoprotein CPY to the lysosome-like vacuole. This has allowed us to identify mutations in eight VPL complementation groups that result in aberrant secretion of up to ∼90\% of the immunoreactive CPY. Other soluble vacuolar proteins are also affected by each vpl mutation, demonstrating that a sorting system for multiple vacuolar proteins exists in yeast. Mislocalized CPY apparently traverses late stages of the secretory pathway, since a vesicle-accumulating sec1 mutation prevents secretion of this protein. Despite the presence of abnormal membrane-enclosed organelles in some of the vpl mutants, maturation and secretion of invertase are not substantially perturbed. Thus vpl mutations define a new class of genes that encode products required for sorting of newly synthesized vacuolar proteins from secretory proteins during their transit through the yeast secretory pathway.},

author = {Rothman, Joel H and Stevens, Tom H},

doi = {10.1016/0092-8674(86)90819-6},

file = {:Users/deepikaa/Library/Application Support/Mendeley Desktop/Downloaded/Rothman, Stevens - 1041 - Protein Sorting in Yeast Mutants Defective in Vacuole Biogenesis Mislocalize Vacuolar Proteins into the Late S.pdf:pdf},

journal = {Cell},

month = {dec},

pages = {1041-1051},

title = {{Protein Sorting in Yeast: Mutants Defective in Vacuole Biogenesis Mislocalize Vacuolar Proteins into the Late, Secretory Pathway}},

volume = {47},

year = {1986}

}

@article{Takei1999a,

abstract = {Amphiphysin, a protein that is highly concentrated in nerve terminals, has been proposed to function as a linker between the clathrin coat and dynamin in the endocytosis of synaptic vesicles. Here, using a cell-free system, we provide direct morphological evidence in support of this hypothesis. Unexpectedly, we also find that amphiphysin-1, like dynamin-1, can transform spherical liposomes into narrow tubules. Moreover, amphiphysin-1 assembles with dynamin-1 into ring-like structures around the tubules and enhances the liposome-fragmenting activity of dynamin-1 in the presence of GTP. These results show that amphiphysin binds lipid bilayers, indicate a potential function for amphiphysin in the changes in bilayer curvature that accompany vesicle budding, and imply a close functional partnership between amphiphysin and dynamin in endocytosis.},

author = {Takei, Kohji and Slepnev, Vladimir I. and Haucke, Volker and {De Camilli}, Pietro},

doi = {10.1038/9004},

file = {:Users/deepikaa/Library/Application Support/Mendeley Desktop/Downloaded/Takei et al. - 1999 - Functional partnership between amphiphysin and dynamin in clathrin-mediatedendocytosis(2).pdf:pdf;:Users/deepikaa/Library/Application Support/Mendeley Desktop/Downloaded/PMVEAEKW/ncb0599{\_}33.html:html},

issn = {1465-7392},

journal = {Nature Cell Biology},

keywords = {Folder - dynamin},

language = {en},

mendeley-tags = {Folder - dynamin},

month = {may},

pages = {33--39},

title = {{Functional partnership between amphiphysin and dynamin in clathrin-mediated endocytosis}},

url = {http://www.nature.com/ncb/journal/v1/n1/full/ncb0599{\_}33.html http://www.nature.com/ncb/journal/v1/n1/pdf/ncb0599{\_}33.pdf},

volume = {1},

year = {1999}

}

@article{Lukehart2013,

abstract = {Recycling of cellular membranes and their constituents plays a role for cell survival and growth. In the budding yeast, there are recycling traffics from early and late endosomal compartments to the late Golgi. Here, we examined a possible role for Vps1, a large GTPase, in the recycling traffic of GFP-Snc1 from early endosomes to the late Golgi. In the absence of Vps1 we observed an aberrant accumulation of GFP-Snc1 puncta in the cytoplasm that we identified as early endosomes. The N-terminal GTPase and the C-terminal GED domains of Vps1 are essential for Vps1's function in Snc1 recycling. Our finding of genetic interactions of VPS1 with genes involved in early endosome-to-Golgi traffic further suggests Vps1 functions as a recycling factor in the membrane traffic. Finally, we provide evidence that the severe accumulation of GFP-Snc1 cytoplasmic puncta in vps1$\Delta$ cells is attributed to a mild defect in the retention of the GARP component Vps51 at the late Golgi, as well as a severe disruption of actin cables.},

author = {Lukehart, Joshua and Highfill, Chad and Kim, Kyoungtae},

doi = {10.1139/bcb-2013-0044},

issn = {0829-8211},

journal = {Biochemistry and Cell Biology},

keywords = {GARP,Snc1,Vps1,early-endosome,endosome pr{\'{e}}coce,recyclage,recycling},

month = {dec},

pages = {455--465},

publisher = { NRC Research Press},

title = {{Vps1, a recycling factor for the traffic from early endosome to the late Golgi}},

url = {http://www.nrcresearchpress.com/doi/10.1139/bcb-2013-0044},

volume = {91},

year = {2013}

}

@article{Myers2016,

abstract = {Membrane remodeling by BAR (Bin, Amphiphysin, RVS) domain-containing proteins, such as endophilins and amphiphysins, is integral to the process of endocytosis. However, little is known about the regulation of endocytic BAR domain activity. We have identified an interaction between the yeast Rvs167 N-BAR domain and calmodulin. Calmodulin-binding mutants of Rvs167 exhibited defects in endocytic vesicle release. In vitro, calmodulin enhanced membrane tubulation and constriction by wild-type Rvs167 but not calmodulin-binding-defective mutants. A subset of mammalian N-BAR domains bound calmodulin, and co-expression of calmodulin with endophilin A2 potentiated tubulation in vivo. These studies reveal a conserved role for calmodulin in regulating the intrinsic membrane-sculpting activity of endocytic N-BAR domains.},

author = {Myers, Margaret D. and Ryazantsev, Sergey and Hicke, Linda and Payne, Gregory S.},

doi = {10.1016/j.devcel.2016.03.012},

issn = {1878-1551},

journal = {Developmental Cell},

keywords = {Animals,Calmodulin,Cell Membrane,Constriction,Folder - cmd1,Liposomes,Microfilament Proteins,Nerve Tissue Proteins,Protein Binding,Protein Structure- Tertiary,Saccharomyces cerevisiae,Saccharomyces cerevisiae Proteins,endocytosis},

language = {ENG},

mendeley-tags = {Animals,Calmodulin,Cell Membrane,Constriction,Folder - cmd1,Liposomes,Microfilament Proteins,Nerve Tissue Proteins,Protein Binding,Protein Structure- Tertiary,Saccharomyces cerevisiae,Saccharomyces cerevisiae Proteins,endocytosis},

month = {apr},

pages = {162--173},

title = {{Calmodulin Promotes N-BAR Domain-Mediated Membrane Constriction and Endocytosis}},

url = {http://www.ncbi.nlm.nih.gov/pubmed/27093085},

volume = {37},

year = {2016}

}

@article{Rothman1990,

abstract = {Members of the Mx protein family promote interferon-inducible resistance to viral infection in mammals and act by unknown mechanisms. We identified an Mx-like protein in yeast and present genetic evidence for its cellular function. This protein, the VPS1 product, is essential for vacuolar protein sorting, normal organization of intracellular membranes, and growth at high temperature, implying that Mx-like proteins are engaged in fundamental cellular processes in eukaryotes. Vps1p contains a tripartite GTP binding motif, which suggests that binding to GTP is essential to its role in protein sorting. Vps1p-specific antibody labels punctate cytoplasmic structures that condense to larger structures in a Golgi-accumulating sec7 mutant; thus, Vps1p may associate with an intermediate organelle of the secretory pathway.},

author = {Rothman, Joel H. and Raymond, Christopher K. and Gilbert, Teresa and O'Hara, Patrick J. and Stevens, Tom H.},

doi = {10.1016/0092-8674(90)90070-U},

file = {:Users/deepikaa/Library/Application Support/Mendeley Desktop/Downloaded/Rothman et al. - 1990 - A putative GTP binding protein homologous to interferon-inducible Mx proteins performs an essential function (2).pdf:pdf},

issn = {0092-8674},

journal = {Cell},

month = {jun},

pages = {1063--1074},

publisher = {Cell Press},

title = {{A putative GTP binding protein homologous to interferon-inducible Mx proteins performs an essential function in yeast protein sorting}},

url = {https://www.sciencedirect.com/science/article/pii/009286749090070U?via{\%}3Dihub},

volume = {61},

year = {1990}

}

@article{Grigliatti1973,

abstract = {Improved methods for rearing and screening large numbers of flies permitted the recovery of 10 mutations exhibiting a reversible temperature-dependent adult paralysis among 1.1 × 106 flies tested. Of the 10 mutations, two were allelie to fara ts, two were alleles in a new locus, stoned (stn), and six fell into a third area, the shibire (shi) locus. Several of the shi alleles cause embryonic, larval and adult paralysis at 29 ° C as well as structural anomalies of various tissues. In addition to the ts mutations, several non-conditional mutations affecting adult movement were recovered.},

author = {Grigliatti, Thomas A and Hall, Linda and Rosenbluth, Raja and Suzuki, David T},

file = {:Users/deepikaa/Library/Application Support/Mendeley Desktop/Downloaded/Grigliatti et al. - 1973 - Temperature-Sensitive Mutations in Drosophila melanogaster XIV. A Selection of Immobile Adults.pdf:pdf},

journal = {Molecular and General Genetics},

pages = {107--114},

title = {{Temperature-Sensitive Mutations in Drosophila melanogaster XIV. A Selection of Immobile Adults}},

url = {https://link.springer.com/content/pdf/10.1007{\%}2FBF00267238.pdf},

volume = {120},

year = {1973}

}

@article{Rooij2010,

author = {Rooij, I. I. S.-d. and Allwood, E. G. and Aghamohammadzadeh, S. and Hettema, E. H. and Goldberg, M. W. and Ayscough, K. R.},

doi = {10.1242/jcs.070508},

file = {:Users/deepikaa/Library/Application Support/Mendeley Desktop/Downloaded/5NJPTW5K-J Cell Sci-2010-Rooij-3496-506.pdf:pdf},

issn = {0021-9533, 1477-9137},

journal = {Journal of Cell Science},

keywords = {Folder - other proteins},

language = {en},

mendeley-tags = {Folder - other proteins},

month = {oct},

pages = {3496--3506},

title = {{A role for the dynamin-like protein Vps1 during endocytosis in yeast}},

url = {http://jcs.biologists.org/cgi/doi/10.1242/jcs.070508},

volume = {123},

year = {2010}

}

@article{Grabs1997,

abstract = {Amphiphysin is an SH3 domain-containing neuronal protein that is highly concentrated in nerve terminals where it interacts via its SH3 domain with dynamin I, a GTPase implicated in synaptic vesicle endocytosis. We show here that the SH3 domain of amphiphysin, but not a mutant SH3 domain, bound with high affinity to a single site in the long proline-rich region of human dynamin I, that this site was distinct from the binding sites for other SH3 domains, and that the mutation of two adjacent amino acids in dynamin I was sufficient to abolish binding. The dynamin I sequence critically required for amphiphysin binding (PSRPNR) fits in the novel SH3 binding consensus identified for the SH3 domain of amphiphysin via a combinatorial peptide library approach: PXRPXR(H)R(H). Our data demonstrate that the long proline-rich stretch present in dynamin I contained multiple SH3 domain binding sites that recognize interacting proteins with high specificity.},

author = {Grabs, D and Slepnev, V I and Songyang, Z and David, C and Lynch, M and Cantley, L C and {De Camilli}, P},

doi = {10.1074/JBC.272.20.13419},

file = {:Users/deepikaa/Library/Application Support/Mendeley Desktop/Downloaded/Grabs et al. - 1997 - The SH3 domain of amphiphysin binds the proline-rich domain of dynamin at a single site that defines a new SH3 bin.pdf:pdf},

issn = {0021-9258},

journal = {The Journal of biological chemistry},

month = {may},

pages = {13419--25},

pmid = {9148966},

publisher = {American Society for Biochemistry and Molecular Biology},

title = {{The SH3 domain of amphiphysin binds the proline-rich domain of dynamin at a single site that defines a new SH3 binding consensus sequence.}},

url = {http://www.ncbi.nlm.nih.gov/pubmed/9148966},

volume = {272},

year = {1997}

}

@article{Antonny2016,

abstract = {The large GTPase dynamin is the first protein shown to catalyze membrane fission. Dynamin and its related proteins are essential to many cell functions, from endocytosis to organelle division and fusion, and it plays a critical role in many physiological functions such as synaptic transmission and muscle contraction. Research of the past three decades has focused on understanding how dynamin works. In this review, we present the basis for an emerging consensus on how dynamin functions. Three properties of dynamin are strongly supported by experimental data: first, dynamin oligomerizes into a helical polymer; second, dynamin oligomer constricts in the presence of GTP; and third, dynamin catalyzes membrane fission upon GTP hydrolysis. We present the two current models for fission, essentially diverging in how GTP energy is spent. We further discuss how future research might solve the remaining open questions presently under discussion.},

author = {Antonny, Bruno and Burd, Christopher and {De Camilli}, Pietro and Chen, Elizabeth and Daumke, Oliver and Faelber, Katja and Ford, Marijn and Frolov, Vadim A and Frost, Adam and Hinshaw, Jenny E and Kirchhausen, Tom and Kozlov, Michael M and Lenz, Martin and Low, Harry H and McMahon, Harvey and Merrifield, Christien and Pollard, Thomas D and Robinson, Phillip J and Roux, Aur{\'{e}}lien and Schmid, Sandra},

doi = {10.15252/embj.201694613},

issn = {0261-4189},

journal = {The EMBO Journal},

keywords = {GTPase,dynamin,endocytosis,membrane fission,molecular motor},

month = {nov},

pages = {2270--2284},

pmid = {27670760},

title = {{Membrane fission by dynamin: what we know and what we need to know}},

url = {http://emboj.embopress.org/lookup/doi/10.15252/embj.201694613},

volume = {35},

year = {2016}

}

@article{Meinecke2013b,

abstract = {Dynamin mediates various membrane fission events, including the scission of clathrin-coated vesicles. Here, we provide direct evidence for cooperative membrane recruitment of dynamin with the BIN/amphiphysin/Rvs (BAR) proteins, endophilin and amphiphysin. Surprisingly, endophilin and amphiphysin recruitment to membranes was also dependent on binding to dynamin due to auto-inhibition of BAR-membrane interactions. Consistent with reciprocal recruitment in vitro, dynamin recruitment to the plasma membrane in cells was strongly reduced by concomitant depletion of endophilin and amphiphysin, and conversely, depletion of dynamin dramatically reduced the recruitment of endophilin. In addition, amphiphysin depletion was observed to severely inhibit clathrin-mediated endocytosis. Furthermore, GTP-dependent membrane scission by dynamin was dramatically elevated by BAR domain proteins. Thus, BAR domain proteins and dynamin act in synergy in membrane recruitment and GTP-dependent vesicle scission.},

author = {Meinecke, Michael and Boucrot, Emmanuel and Camdere, Gamze and Hon, Wai-Ching and Mittal, Rohit and McMahon, Harvey T},

doi = {10.1074/jbc.M112.444869},

file = {:Users/deepikaa/Library/Application Support/Mendeley Desktop/Downloaded/Meinecke et al. - 2013 - Cooperative recruitment of dynamin and BINamphiphysinRvs (BAR) domain-containing proteins leads to GTP-depen(2).pdf:pdf},

issn = {1083-351X},

journal = {The Journal of biological chemistry},

month = {mar},

pages = {6651--61},

pmid = {23297414},

publisher = {American Society for Biochemistry and Molecular Biology},

title = {{Cooperative recruitment of dynamin and BIN/amphiphysin/Rvs (BAR) domain-containing proteins leads to GTP-dependent membrane scission.}},

url = {http://www.ncbi.nlm.nih.gov/pubmed/23297414 http://www.pubmedcentral.nih.gov/articlerender.fcgi?artid=PMC3585104},

volume = {288},

year = {2013}

}
@article{Gurunathan2002,
abstract = {Yeast produce two classes of secretory vesicles (SVs) that differ in both density and cargo protein content. In late-acting secretory mutants (e.g. snc1(ala43) and sec6-4), both low- (LDSV) and high-density (HDSV) classes of vesicles accumulate at restrictive temperatures. Here, we have found that disruptions in the genes encoding a dynamin-related protein (VPS1) or clathrin heavy chain (CHC1) abolish HDSV production, yielding LDSVs that contain all secreted cargos. Interestingly, disruption of the PEP12 gene, which encodes the t-SNARE that mediates all Golgi to pre-vacuolar compartment (PVC) transport, also abolishes HDSV production. In contrast, deletions in genes that selectively confer vacuolar hydrolase sorting to the PVC or protein transport to the vacuole (i.e. VPS34 and VAM3, respectively) have no effect. Thus, one branch of the secretory pathway in yeast involves an intermediate sorting compartment and has a specific requirement for clathrin and a dynamin-related protein in SV biogenesis.},
author = {Gurunathan, Sangiliyandi and David, Doris and Gerst, Jeffrey E},
doi = {10.1093/EMBOJ/21.4.602},
file = {:Users/deepikaa/Library/Application Support/Mendeley Desktop/Downloaded/Gurunathan, David, Gerst - 2002 - Dynamin and clathrin are required for the biogenesis of a distinct class of secretory vesicles in yeas.pdf:pdf},
issn = {0261-4189},
journal = {The EMBO journal},
month = {feb},
number = {4},
pages = {602--14},
pmid = {11847108},
publisher = {European Molecular Biology Organization},
title = {{Dynamin and clathrin are required for the biogenesis of a distinct class of secretory vesicles in yeast.}},
url = {http://www.ncbi.nlm.nih.gov/pubmed/11847108 http://www.pubmedcentral.nih.gov/articlerender.fcgi?artid=PMC125853},
volume = {21},
year = {2002}
}
